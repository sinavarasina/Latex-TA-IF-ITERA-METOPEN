% !TeX root = thesis.tex
%--------------------------------------------------------------------%
%
% Template TA LaTeX Teknik Informatika ITERA.
% Editor: Radhinka Bagaskara, Martin C.T. Manullang, I Wayan Wiprayoga Wisesa, Jose Alfredo Sitanggang (IF 2016), Ardoni Yeriko Rifana Gultom (IF 2021), Syabana Minggus Noviantosa (IF 2018), Rizki Alfaina (IF 2021)
% Version 2025.5
% TELAH DISESUAIKAN DENGAN FORMAT PERPUSTAKAAN ITERA MEI 2025 (UNESCO FORMAT)
%
% Berdasarkan "Templat LaTeX Tesis Informatika ITB" oleh Petra Barus & Peb Ruswono Aryan
% https://github.com/petrabarus/if-itb-latex
% Berdasarkan Panduan Tugas Akhir Teknik Informatika ITERA v1.3
% https://docs.google.com/document/d/1SYtSpRevbRvscXIJRAxuT41kJqqzsHyw/edit?usp=sharing&ouid=103935211052656359121&rtpof=true&sd=true 
%
%--------------------------------------------------------------------%
%
% Berkas ini berisi struktur utama dokumen LaTeX yang akan dibuat.
%
%--------------------------------------------------------------------%

% Set jenis dokumen Tugas Akhir
\documentclass[article]{report} % Untuk versi perpustakaan
\usepackage[paperwidth=155mm, % Setting layout kertas UNESCO (155 x 230 mm)
			paperheight=230mm,
			top=2cm, 
			bottom=2cm, 
			left=2cm,
			right=2cm, 
			includefoot, 
			heightrounded, 
			bindingoffset=0.5cm]
			{geometry} 

%-------------------------------------------------------------------%
%
% Konfigurasi dokumen LaTeX untuk laporan tesis IF ITB
% 
%
% @author Radhinka Bagaskara, Petra Novandi (ITB)
%
%-------------------------------------------------------------------%
%
% Berkas asli berasal dari Steven Lolong
%
%-------------------------------------------------------------------%

% Import package penting
\usepackage[utf8]{inputenc}
\usepackage{subcaption} % Paket untuk mengatur gambar berdampingan
\usepackage{graphicx}
\usepackage{titling}
\usepackage{blindtext} % Untuk lorem ipsum
\usepackage{sectsty} % Untuk header & judul
\usepackage{chngcntr} % Untuk penambahan nomor caption
\usepackage{etoolbox} % Untuk CRUD variabel (?)
\usepackage{array} % % Untuk tabel di math mode
\usepackage{float} % Untuk tabular
\usepackage{longtable} % Untuk tabel yang potong halaman
\usepackage{amsmath} % Untuk equation
\usepackage{enumitem} % Untuk list enumerate yg lebih rapi
\usepackage{nameref} % Untuk dapat menggunakan nameref, referensi dengan nama section/caption
\usepackage[bookmarks]{hyperref}
\usepackage{lipsum}
\hypersetup{
	colorlinks,
	citecolor=black,
	filecolor=black,
	linkcolor=black,
	urlcolor=black
}

% Setting indensasi
\usepackage{indentfirst}
\usepackage{parskip}
\usepackage[indentafter]{titlesec}
\usepackage{microtype}
%\setlength{\parindent}{20pt}
\setlength{\parskip}{0pt}
\newlength{\sectionindent}
\settowidth{\sectionindent}{\thesection\hspace{0.965em}} % Agar indensasi paragraf selaras dengan indensasi judul subbab (Alfaina)
\setlength{\parindent}{\sectionindent}

% Linespacing 1.5. Tidak serupa dengan 1.5 di Word (RDB)
\renewcommand{\baselinestretch}{1.5}

% Agar tidak ada kata yang terpotong setiap baris kalimat
\hyphenpenalty=10000

% Font
%\usepackage{mathptmx} 
\usepackage{newtx} 
% Times New Roman itu copyright dari Microsoft. Ini alternatifnya (RDB)

% Judul bahasa Indonesia
\usepackage[bahasa]{babel}

% Format tanggal
\usepackage[style=ddmmyyyy,datesep=-]{datetime2}

% Format citation
\usepackage[backend=bibtex,citestyle=ieee]{biblatex} % Untuk bisa jalan di Texstudio, harus backend=bibtex
% Seriusan, JANGAN GANTI BACKENDNYA (RDB)

% Remove "In:" before journal titles
\renewbibmacro{in:}{}

% Ensure URLs, DOIs, and ISSNs use the default font (e.g., Times New Roman)
\renewcommand*{\UrlFont}{\rmfamily} % Use default font for URLs
\DeclareFieldFormat{issn}{#1}       % Use default font for ISSN
\DeclareFieldFormat{doi}{#1}        % Use default font for DOI

% Remove DOI and URL fields if not needed (optional)
\AtEveryBibitem{
  \clearfield{doi}
  \clearfield{url}
}

\DeclareLanguageMapping{bahasa}{english}

% Package untuk link di daftar isi.
%\usepackage{titlesec}       % Package Format judul
%\usepackage{parskip}
\usepackage{ragged2e}		% Alignment
\usepackage{multirow}		% Untuk bisa merge cell di tabel
\usepackage{tikz}			% Untuk menggambar kotak pas foto
\usepackage{setspace}		% Spacing paragraph
\usepackage{fancyhdr}		% Agar nomor halaman di pojok kanan atas
\usepackage{caption} 		% Caption gambar & tabel
\usepackage{comment}

\captionsetup[figure]{font=footnotesize} % Mengatur caption gambar ke 10pt (footnotesize)
\captionsetup[figure]{labelsep=space} % Mengubah ":" menjadi spasi setelah nomor gambar
\captionsetup[table]{font=footnotesize}
\captionsetup[table]{labelsep=space}
\captionsetup[equation]{font=footnotesize}
\captionsetup[lstlisting]{font=footnotesize, labelsep=space}

% Setting supaya nomor halaman pertama dengan "chapter"
% berada di kanan atas
\fancypagestyle{plain}{%
	\fancyhf{}%
	\renewcommand{\headrulewidth}{0pt}
	\fancyhead[R]{\thepage}
}

% Fancy header selang-seling kiri-kanan atas (Ardoni)
\fancypagestyle{alternatingstyle}{
  \fancyhf{} % Clear all headers and footers
  \renewcommand{\headrulewidth}{0pt}
  \fancyhead[RO]{\thepage} % Right Odd
  \fancyhead[LE]{\thepage} % Left Even
}

\newcommand{\forceoddpage}{
  \ifodd\value{page}
  \else
    \hbox{}
    \newpage
  \fi
}

% Setting judul
\chapterfont{\centering \large}
\titleformat{\chapter}[display]%
  	{\large\centering\bfseries}%
  	{\chaptertitlename\ \thechapter}{0pt}%
  	{\large\bfseries\uppercase}
\titleformat{\section}%
	{\normalfont\normalsize\bfseries}{\thesection}{1em}{}
\titleformat{\subsection}%
	{\normalfont\normalsize\bfseries}{\thesubsection}{1em}{}

%\makeatletter
%\def\@makechapterhead#1{%
%  \vspace*{-2cm} % GESER KE ATAS (coba-coba sampai pas di margin)
%  {\parindent \z@ \centering
%   \normalfont
%   \interlinepenalty\@M
%   \Large\bfseries \MakeUppercase{\@chapapp\space \thechapter}\par\nobreak
%   \vskip 0pt
%   \Large\bfseries \MakeUppercase{#1}\par\nobreak
%   \vskip 20pt
%  }}
%\makeatother

    
% Setting spacing di setiap judul chapter
\titlespacing*{\chapter}{0pt}{0pt}{10pt}
\titlespacing*{\section}{0pt}{10pt}{0pt}
\titlespacing*{\subsection}{0pt}{10pt}{0pt}
\titlespacing*{\subsubsection}{0pt}{10pt}{0pt}

\usepackage{listings}
\usepackage{xcolor}
\usepackage{geometry}


% Setting nomor pada subbsubsubbab
\setcounter{secnumdepth}{3}

% Counter untuk figure dan table, agar bertambah walaupun lintas subbab
\counterwithout{figure}{section}
\counterwithout{table}{section}

% Supaya tidak ada garis di header
\renewcommand{\headrulewidth}{0pt}

% Setting daftar isi, daftar gambar, daftar tabel, daftar rumus
\usepackage[titles]{tocloft}
\setlength{\cftbeforechapskip}{5.2pt}
% ------------------------------------------------------------------
% Make the vertical gap before *every* figure/table entry the same:
\setlength{\cftbeforefigskip}{0pt}   % or use whatever fixed length you like
\setlength{\cftbeforetabskip}{0pt}
% ------------------------------------------------------------------
\cftsetindents{section}{1.5em}{2.3em}
\cftsetindents{subsection}{3em}{3em}
\cftsetindents{subsubsection}{4.5em}{4em}
\setlength{\cfttabindent}{1.5em}
\setlength{\cftfigindent}{1.5em}

\usepackage{etoolbox}
\makeatletter
  % before starting any list: make addvspace do nothing
  \pretocmd{\listoffigures}{\begingroup\renewcommand*{\addvspace}[1]{}}{}{}
  \apptocmd {\listoffigures}{\endgroup}{}{}
  \pretocmd{\listoftables} { \begingroup\renewcommand*{\addvspace}[1]{}}{}{}
  \apptocmd {\listoftables} { \endgroup}{}{}
\makeatother

\setcounter{tocdepth}{3} 

% Tambahkan kata "BAB" sebelum nomor bab di daftar isi
\renewcommand*\cftchappresnum{\MakeUppercase{BAB}~}
\renewcommand\chaptername{BAB}
\settowidth{\cftchapnumwidth}{\cftchappresnum}
\renewcommand{\cftchapaftersnumb}{\quad}
\addtocontents{toc}{
	\protect\renewcommand*\protect\cftchappresnum{\MakeUppercase{\chaptername}~}
}

\renewcommand{\cftchapleader}{\dotfill} 
\renewcommand{\cftsecleader}{\dotfill}
\renewcommand{\cftsubsecleader}{\dotfill}
\renewcommand{\cftsubsubsecleader}{\dotfill}
\renewcommand{\cftfigleader}{\dotfill}
\renewcommand{\cfttableader}{\dotfill}

% Nama daftar isi, gambar, tabel, rumus dalam bahasa Indonesia
\addto\captionsbahasa{%
	\renewcommand{\contentsname}{DAFTAR ISI}%
	\renewcommand{\listfigurename}{DAFTAR GAMBAR}%
	\renewcommand{\listtablename}{DAFTAR TABEL}%
}

% Setting daftar rumus
\newcommand{\listequationsname}{DAFTAR RUMUS}
\newlistof{myequations}{equ}{\listequationsname}
\newcommand{\myequations}[1]{
	\addcontentsline{equ}{myequations}{\protect\numberline{\theequation}#1}
}

% Mengatur format daftar rumus
\renewcommand{\cftmyequationspresnum}{}
\newlength{\mylenf}
\settowidth{\mylenf}{\cftmyequationspresnum}
\setlength{\cftmyequationsnumwidth}{\dimexpr\mylenf+5em} % Menyesuaikan nomor
\setlength{\cftmyequationsindent}{1.5em} % Menambahkan indentasi daftar rumus
\renewcommand{\cftmyequationsleader}{\dotfill}

% Setting penulisan rumus yang baru (Ardoni)
\newenvironment{equationcaptioned}[3][]{%
  \refstepcounter{equation}
  \myequations{#3}
  \begin{center} % TODO: Rumus harusnya rata kiri
    \vspace{1em}
    % Removed the fcolorbox and kept only the minipage
    \begin{minipage}{0.9\linewidth}
		\begin{spacing}{0.7} % ⬅️ Spasi 1.0 hanya di rumus
			\begin{equation*}
				#2
				\ifstrempty{#1}{}{\label{#1}}
			\end{equation*}
		\end{spacing}
		\begin{flushright}
			(\footnotesize\theequation)
		\end{flushright}
    \end{minipage}
  \end{center}
  \vspace{0.5em}
}{}

% Setting daftar gambar dan daftar tabel
\renewcommand\cftfigpresnum{Gambar\ }
\renewcommand\cfttabpresnum{Tabel\ }

\settowidth{\mylenf}{\cftfigpresnum}
\setlength{\cftfignumwidth}{\dimexpr\mylenf+1.5em}
\setlength{\cfttabnumwidth}{\dimexpr\mylenf+0.5em}

% Setting penomoran caption gambar, tabel, dan rumus
\renewcommand{\thefigure}{\arabic{chapter}.\arabic{figure}}
\renewcommand{\thetable}{\arabic{chapter}.\arabic{table}}
\renewcommand\theequation{Rumus~\arabic{chapter}.\arabic{equation}} % Penomoran rumus berdasarkan Babnya (Minggus)

\usepackage{listings}
\usepackage{xcolor}
\usepackage{geometry}

% Pengaturan untuk penulisan blok kode (listings) (Ardoni)
\lstset{
    language=Python,                  	% Bahasa pemrograman
    basicstyle=\ttfamily\scriptsize,	% Gaya font kode (diperkecil)
    numbers=left,                      	% Menampilkan nomor baris di sisi kiri
    numberstyle=\tiny\color{black},		% Warna nomor baris
    showstringspaces=false,            	% Tidak menampilkan icon spasi dalam string
    frame=single,                      	% Membuat frame di sekitar kode
    rulecolor=\color{black},           	% Warna garis bingkai
    breaklines=true,                   	% Membungkus kode jika lebih panjang dari satu baris
    breakatwhitespace=true,           	% Membungkus di spasi
    backgroundcolor=\color{white},    	% Warna latar belakang
    commentstyle=\color{green}, 		% Warna untuk teks comment
    keywordstyle=\color{blue}, 			% Warna untuk teks fungsi
    stringstyle=\color{red},			% Warna untuk teks string
    keepspaces=true,                   	% Menjaga spasi di dalam kode
    xleftmargin=10pt,                 	% Margin kiri
    xrightmargin=10pt,                 	% Margin kanan
    aboveskip=0pt,                    	% Jarak di atas kode
    belowskip=0pt,                    	% Jarak di bawah kode
    lineskip=-1pt,
    extendedchars=true                 	% Mengizinkan karakter non-ASCII
    escapeinside={\%*}{*)}            	% Escape karakter LaTeX dalam listings
}

% Daftar Kode
\usepackage{tocloft}
\newcommand{\listoflistingscaption}{DAFTAR KODE}
\renewcommand{\lstlistingname}{Kode}
\renewcommand{\lstlistlistingname}{\listoflistingscaption}

% Menentukan format penomoran untuk listings dengan angka Arab
\makeatletter
\renewcommand{\thelstnumber}{\@arabic\c@lstnumber}
\makeatother

% Menetapkan counter listings agar mengikuti chapter dan menggunakan format "chapter.listing"
\makeatletter
\@removefromreset{lstlisting}{chapter}  % Mencegah reset counter listing saat chapter berubah
\makeatother
\renewcommand{\lstlistingname}{Kode}
\AtBeginDocument{\renewcommand{\thelstlisting}{\arabic{chapter}.\arabic{lstlisting}}}

% Mengatur prefix "Kode" di daftar kode dengan nomor Arab
\renewcommand{\lstlistoflistings}{%
  \begingroup
  \let\oldnumberline\numberline
  \renewcommand{\numberline}[1]{Kode~##1~}
  \listof{lstlisting}{\listoflistingscaption}
  \endgroup
}
% Mengatur titik-titik pada daftar kode agar sama persis dengan daftar rumus
\makeatletter
% Definisikan kembali cara menampilkan daftar kode
\renewcommand*\l@lstlisting[2]{%
  \vskip \cftbeforelstlistingskip
  \@dottedtocline{1}{1.5em}{6em}{#1}{%
    \leaders\hbox{%
      $\m@th \mkern \@dotsep mu \hbox{.}\mkern \@dotsep mu$%
    }\hfill\nobreak#2%
  }%
}

% Pastikan nilai @dotsep sama dengan yang digunakan di daftar rumus
\renewcommand\@dotsep{2.5}  % Coba nilai yang berbeda: 1.5, 2.0, 2.5, 3.0, dll.
\makeatother

\usepackage{hyperref}
\hypersetup{
    colorlinks=true,
    linkcolor=black,   % warna untuk \ref, \autoref, dll
    urlcolor=blue,     % warna untuk \href (tautan web)
    citecolor=black    % warna untuk \cite
}

\usepackage{cleveref}
% Setelah paket cleveref dimuat
\crefalias{lstlisting}{kode}
\crefname{kode}{Kode}{Kode}
\Crefname{kode}{Kode}{Kode}
\newcommand{\coderef}[1]{Kode~\ref{#1}}

% Untuk mengganti nama bulan di babel bahasa
% tapi belum jalan (RDB)
\StartBabelCommands*{bahasa}{date}
\SetStringLoop{month#1name}{%
	Januari,Februari,Maret,April,Mei,Juni,%
	Juli,Agustus,September,Oktober,November,%
	Desember}
\EndBabelCommands     

% english title
\providecommand\titleEN[1]{\providecommand\thetitleEN{#1}}

% Saya lupa ini buat apa (RDB)
%\renewcommand{\theHsection}{\thepart.section.\thesection}

% Semua dibawah ini berhubungan dengan CRUD variabel (ada simbol @)
\makeatletter % Jangan dihapus

% Command untuk variabel NIM
\newcommand{\nim}[1]{\def\@nim{#1}}
\newcommand{\printnim}{\@nim}

% Command untuk variabel Dosen Pembimbing I & II
\newcommand{\namadosbinga}[1]{\def\@namadosbinga{#1}}
\newcommand{\namadosbingb}[1]{\def\@namadosbingb{#1}}
\newcommand{\nipdosbinga}[1]{\def\@nipdosbinga{#1}}
\newcommand{\nipdosbingb}[1]{\def\@nipdosbingb{#1}}
\newcommand{\printnamadosbinga}{\@namadosbinga}
\newcommand{\printnamadosbingb}{\@namadosbingb}
\newcommand{\printnipdosbinga}{\@nipdosbinga}
\newcommand{\printnipdosbingb}{\@nipdosbingb}
\newcommand{\dosbingA}[2]{\namadosbinga{#1} \nipdosbinga{#2}}
\newcommand{\dosbingB}[2]{\namadosbingb{#1} \nipdosbingb{#2}}

% Command untuk variabel Dosen Penguji I & II
\newcommand{\namapengujia}[1]{\def\@namapengujia{#1}}
\newcommand{\namapengujib}[1]{\def\@namapengujib{#1}}
\newcommand{\nippengujia}[1]{\def\@nippengujia{#1}}
\newcommand{\nippengujib}[1]{\def\@nippengujib{#1}}
\newcommand{\printnamapengujia}{\@namapengujia}
\newcommand{\printnamapengujib}{\@namapengujib}
\newcommand{\printnippengujia}{\@nippengujia}
\newcommand{\printnippengujib}{\@nippengujib}
\newcommand{\pengujiA}[2]{\namapengujia{#1} \nippengujia{#2}}
\newcommand{\pengujiB}[2]{\namapengujib{#1} \nippengujib{#2}}

% Command untuk variabel tanggal dengan format seperti "27 Juli 2025"
\newcommand{\todayIndo}{\the\day~\ifcase\month
	\or Januari\or Februari\or Maret\or April\or Mei\or Juni%
	\or Juli\or Agustus\or September\or Oktober\or November\or Desember\fi~\number\year}

% Command untuk merubah spacing equation
\g@addto@macro\normalsize{%
	\setlength\abovedisplayskip{-10pt}
	\setlength\belowdisplayskip{-10pt}
	\setlength\abovedisplayshortskip{-10pt}
	\setlength\belowdisplayshortskip{-10pt}
}

\makeatother % Jangan dihapus

 % Import konfigurasi laporan
\bibliography{references.bib} % Import list Daftar Pustaka

\begin{document}

    %----------------------------------------------------------------%
    % Konfigurasi Informasi Tugas Akhir
    %----------------------------------------------------------------%
    
    % Judul Tugas Akhir, dalam Bahasa Indonesia	
    \title{Perancangan dan Implementasi Framework Desktop Environment Berbasis Entity-Component-System (ECS) pada Wayland}
    % Judul Tugas Akhir dalam Bahasa Inggris
    \titleEN{Design and Implementation of an Entity-Component-System (ECS)-Based Desktop Environment Framework on Wayland}
    % JUDUL AKAN DITULIS DALAM HURUF KAPITAL; Font size 16 pt; Bold; Tidak melebihi 4 baris
    
    % Informasi Mahasiswa
	\author{Varasina Farmadani}		% Nama Mahasiswa
	\nim{123140107}				% NIM Mahasiswa
	
	%Informasi Dosen Pembimbing
	\dosbingA%
		{Dosen Pembimbing I}%			% Nama Dosen Pembimbing 1
		{NIP. 19900000 2000 00 0 000}	% NIP Dosen Pembimbing 1
	\dosbingB%
		{Dosen Pembimbing II}%			% Nama Dosen Pembimbing 2
		{NIP. 19900000 2000 00 0 000}	% NIP Dosen Pembimbing 2
		
	%Informasi Dosen Penguji
	\pengujiA%
		{Dosen Penguji I}%				% Nama Dosen Penguji 1
		{NIP. 19900000 2000 00 0 000}	% NIP Dosen Penguji 1
	\pengujiB%
		{Dosen Penguji II}%				% Nama Dosen Penguji 2
		{NIP. 19900000 2000 00 0 000}	% NIP Dosen Penguji 2

	\sloppy % mencegah text overflow. (Jose)
    \pagenumbering{roman}
    \setcounter{page}{1} % Nomor halaman dimulai dengan "ii" di hal. Pengesahan

	%----------------------------------------------------------------%
	% Konten Laporan
	%----------------------------------------------------------------%
	\input{chapters/cover-hard} 	% Hardcover
	\clearpage
\pagestyle{fancy}
\fancyhf{}
\fancyhead[R]{\thepage}
\phantomsection% 
\addcontentsline{toc}{chapter}{LEMBAR PENGESAHAN}

\begin{center}

	\large \bfseries \MakeUppercase{Lembar Pengesahan}
    
    \small \normalfont \singlespacing \justify{
    Saya menyatakan bahwa Tugas Akhir berjudul “{\thetitle}" merupakan hasil karya saya sendiri dan belum pernah diajukan, baik sebagian maupun seluruhnya, di Institut Teknologi Sumatera atau institusi pendidikan lain oleh saya maupun pihak lain.}
    %Tugas Akhir Sarjana dengan judul "{\thetitle}" adalah benar dibuat oleh saya sendiri dan belum pernah dibuat dan diserahkan sebelumnya, baik sebagian ataupun seluruhnya, baik oleh saya ataupun orang lain, baik di Institut Teknologi Sumatera maupun di institusi pendidikan lainnya.

	% Informasi Mahasiswa
    \flushleft
	\setlength{\tabcolsep}{0pt}
	\begin{tabular}{p{0.59\textwidth} p{0.3\textwidth}}
        \vspace{0.1cm}
		Lampung Selatan, \todayIndo & %
		\multirow{6}{*}{
			% Kotak pasfoto 3x4
			\phantom{----------------------} % Amazing hack biar kotaknya ke kanan (RDB)
			\begin{tikzpicture}
				\draw rectangle (2cm,3cm) node[pos=0.5] {Foto 2x3};
			\end{tikzpicture}
		}\\
		Penulis, \\
		& \\
		& \\
		%& \\
		\theauthor\\
		NIM. \printnim
	\end{tabular}
	% Informasi Dosen
	\vspace{0.4cm}
        \begin{center}
        Diperiksa dan disetujui oleh,
        \end{center}
        \vspace{0.1cm}

	\justify
    \setlength{\tabcolsep}{0pt}
    \begin{tabular}{ m{0.5cm}  m{0.7\textwidth} >{\centering\arraybackslash}m{0.3\textwidth}}
        \multicolumn{2}{c}{\hspace*{70pt}Pembimbing} & \multicolumn{1}{c}{} \\[2pt]
		1. & \printnamadosbinga & \\
		 & \printnipdosbinga & ............... \\%[4pt]
		 & \\
		2. & \printnamadosbinga & \\
		& \printnipdosbinga & ............... \\%[4pt]
		& \\
		\multicolumn{2}{c}{\hspace*{70pt}Penguji} & \multicolumn{1}{c}{} \\[2pt]
		1. & \printnamapengujia & \\
		& \printnippengujia & ............... \\[4pt]
            %& \\
		2. & \printnamapengujib & \\
		& \printnippengujib & ............... \\
    \end{tabular}
%	\vfill

	\begin{center}
		\fontsize{10pt}{10pt}
        \vspace{0.45cm}
		Disahkan oleh,\\
		Koordinator Program Studi Teknik Informatika\\
		Fakultas Teknologi Industri\\
		Institut Teknologi Sumatera
		\vspace{1.8cm}\\
		Andika Setiawan, S.Kom., M.Cs. \\ % TODO: make automatic
		NIP. 19911127 2022 03 1 007 \\
	\end{center}
	\vfill

\end{center}
\clearpage 		% Lembar Pengesahan
    %----------------------------------------------------------------%
    % Aktivasi twoside (agar ada margin gutter 0.5 cm) & nomor halaman selang-seling
    % mulai dari setelah Lembar Pengesahan (Ardoni)
    \clearpage
	\newgeometry{
		top=2cm,
		bottom=2cm,
		left=2cm,
		right=2cm,
		includefoot,
		heightrounded,
		bindingoffset=0.5cm,
		twoside % Agar halaman diset akan bolak-balik
	}
    \pagestyle{alternatingstyle}
    %----------------------------------------------------------------%
	\input{chapters/statement} 		% Halaman Pernyataan Orisinalitas
	\clearpage
\phantomsection% 
\addcontentsline{toc}{chapter}{HALAMAN PERSETUJUAN PUBLIKASI}

\begin{center}
	% \small % Add this line to make the font smaller
	\smallskip
	
	\normalsize \bfseries \MakeUppercase{
		HALAMAN PERNYATAAN PERSETUJUAN PUBLIKASI \\
		TUGAS AKHIR UNTUK KEPENTINGAN AKADEMIS
	}\linebreak
	
	\normalfont \onehalfspacing \justifying % Changed \normalsize to \small
	Sebagai civitas akademik Institut Teknologi Sumatera, saya yang bertanda tangan di bawah ini:
	
	\flushleft
	\setlength{\tabcolsep}{0pt}
	\begin{tabular}{l l}
		Nama 			&  : \theauthor\\
		NIM 			&  : \printnim\\
		Program Studi \	&  : Teknik Informatika\\
		Fakultas 		&  : Teknologi Industri\\
		Jenis Karya 	&  : Tugas Akhir\\
	\end{tabular}

	\justifying
	\noindent demi pengembangan ilmu pengetahuan, menyetujui untuk memberikan kepada Institut Teknologi Sumatera \textbf{Hak Bebas Royalti Noneksklusif \textit{(Non-exclusive Royalty Free Right)}} atas karya ilmiah saya yang berjudul:
	
	\centering
	\textbf{\thetitle}
	
	\justifying
	beserta perangkat yang ada (jika diperlukan). Dengan Hak Bebas Royalti Noneksklusif ini Institut Teknologi Sumatera berhak menyimpan, mengalihmedia/formatkan, mengelola dalam bentuk pangkalan data (\textit{database}), merawat, dan memublikasikan tugas akhir saya selama tetap mencantumkan nama saya sebagai penulis/pencipta dan sebagai pemilik Hak Cipta.
	
	Demikian pernyataan ini saya buat dengan sebenarnya. \\
	
	\centering
	Dibuat di : Lampung Selatan\\
	Pada tanggal : \todayIndo\\ % Format bulan harusnya nama panjang, belum kepikiran gimana caranya (RDB)
	\bigskip
	Yang menyatakan\\
	\vspace{1.5cm}
	\theauthor
	
\end{center}
\clearpage
 	% Halaman Persetujuan Publikasi
	\input{chapters/forewords} 		% Kata Pengantar
	\input{chapters/summary} 		% Ringkasan
	\input{chapters/abstract-id} 	% Abstrak (Indonesia)
	\input{chapters/abstract-en} 	% Abstrak (Inggris)    
    %----------------------------------------------------------------%
    % Halaman Daftar
    %----------------------------------------------------------------%    
	% Daftar Isi
	\phantomsection% 
	\addcontentsline{toc}{chapter}{DAFTAR ISI}
	\tableofcontents
	% Daftar tabel
	\phantomsection% 
	\addcontentsline{toc}{chapter}{DAFTAR TABEL}
	\listoftables
	% Daftar gambar
	\phantomsection% 
	\addcontentsline{toc}{chapter}{DAFTAR GAMBAR} % TODO: No halaman ga selang-seling di Daftar? (RDB)
	\listoffigures
	% Daftar rumus
	% \pagestyle{daftarrumusstyle}
	\phantomsection% 
	\addcontentsline{toc}{chapter}{DAFTAR RUMUS}
	\listofmyequations
	% \thispagestyle{daftarrumusstyle}
	% \pagebreak
	%    \input{chapters/symbols}
	% Daftar kode
	\phantomsection% 
	\addcontentsline{toc}{chapter}{DAFTAR KODE}
	\lstlistoflistings
	\pagebreak
    %----------------------------------------------------------------%
    % Konfigurasi Bab
    %----------------------------------------------------------------%
    \renewcommand{\chaptername}{BAB}
    % Bab: Arabic
    \renewcommand{\thechapter}{\Roman{chapter}}
    % Sub-bab: Roman
    \renewcommand\thesection{\arabic{chapter}.\arabic{section}}
    % Setting supaya nomor halaman pertama dengan "chapter"
    % berada di tengah bawah, tapi selanjut2nya di kanan atas
    \fancypagestyle{plain}{%
    	\fancyhf{}%
    	\renewcommand{\headrulewidth}{0pt}
    	\fancyhead[]{}
    	\fancyfoot[C]{\thepage}
    }
    % Reset penomoran halaman menjadi 1
    \setcounter{page}{1}
    \pagenumbering{arabic}    
    \justifying
    % Mulai selang-seling halaman
    \forceoddpage
    \cleardoublepage
    \pagestyle{alternatingstyle}
    %----------------------------------------------------------------%
    % Daftar Bab
    % Untuk menambahkan daftar bab, buat berkas bab misalnya `chapter-6` di direktori `chapters`, dan masukkan ke sini.
    %----------------------------------------------------------------%
    \newpage
\chapter{PENDAHULUAN} \label{Bab I}

\section{Latar Belakang} \label{I.Latar Belakang}
Perkembangan sistem operasi modern, khususnya berbasis Linux, mendorong kebutuhan akan lingkungan desktop (\textit{desktop environment}) yang lebih efisien, modular, dan dapat beradaptasi dengan teknologi grafis baru seperti Wayland. Wayland merupakan protokol komunikasi antara \textit{compositor} dan aplikasi klien yang menggantikan X11 sebagai standar tampilan modern. Dengan desain yang sederhana, efisien, dan berbasis objek pada tataran protokol, Wayland memberi ruang bagi pengembang untuk membangun sistem tampilan dengan latensi rendah dan arsitektur modular. \par

Dalam praktiknya, ekosistem \textit{desktop environment} arus utama masih bertumpu pada kerangka berparadigma objek atau modular tradisional. GNOME (Mutter/GTK) menggunakan \textit{GObject} sebagai sistem objek di C, KDE Plasma/KWin berbasis Qt (QObject dan \textit{signals/slots}), sedangkan di ranah kompositor Wayland, Sway memanfaatkan \textit{wlroots} (pustaka modular di C). Hyprland ditulis dalam C++ mengintegrasikan Aquamarine sebagai \textit{rendering backend}. Pendekatan tersebut efektif dan matang, namun umumnya belum mengadopsi arsitektur \textit{Entity-Component-System} (ECS). \par

Di sisi lain, paradigma \textit{Data-Oriented Design} (DOD) yang diimplementasikan melalui arsitektur ECS menawarkan penataan data yang lebih bersahabat terhadap \textit{cache} dan paralelisasi, sehingga berpotensi meningkatkan efisiensi jalur eksekusi kritis (misalnya manajemen jendela, komposisi frame, dan penanganan input). Pendekatan ini telah terbukti efektif dalam dunia \textit{game engine} (misal Unity/Bevy), tetapi penerapannya pada sistem \textit{desktop environment} dan kompositor Wayland masih terbatas. \par

Wayland sebagai protokol tidak mengikat arsitektur internal kompositor; karena itu, terdapat peluang penelitian untuk merancang dan mengimplementasikan sebuah \textit{framework} \textit{desktop environment} berbasis ECS di atas Wayland. Framework semacam ini ditujukan untuk mengelola entitas seperti jendela/surface, perangkat input, dan objek render secara efisien, sekaligus menjaga modularitas agar mudah diperluas. Dengan demikian, penelitian ini memposisikan ECS/DOD bukan sebagai klaim pengganti OOP secara umum, melainkan sebagai eksplorasi arsitektur alternatif pada domain \textit{desktop environment} modern. \par

\section{Rumusan Masalah} \label{I.Rumusan Masalah}
Berdasarkan latar belakang tersebut, maka rumusan masalah dalam penelitian ini adalah sebagai berikut: \par
\begin{enumerate}[noitemsep]
    \item Bagaimana merancang arsitektur \textit{framework} \textit{desktop environment} berbasis \textit{Entity-Component-System} (ECS) yang dapat beroperasi di atas protokol Wayland? 
    \item Bagaimana mengimplementasikan \textit{framework} tersebut sehingga mampu mengelola entitas sistem seperti jendela/surface, input, dan rendering secara efisien? 
    \item Bagaimana mengevaluasi performa dan efisiensi \textit{framework} yang dirancang terhadap pemanfaatan sumber daya sistem (memori, latensi input, dan waktu render)? 
\end{enumerate}
\par

\section{Tujuan Penelitian} \label{I.Tujuan}
Tujuan dari penelitian ini adalah sebagai berikut: \par
\begin{enumerate}[noitemsep]
    \item Merancang arsitektur \textit{framework} \textit{desktop environment} berbasis \textit{Entity-Component-System} (ECS) di atas protokol Wayland. 
    \item Mengimplementasikan prototipe \textit{framework} yang mendukung pengelolaan entitas utama seperti jendela/surface, perangkat input, dan sistem rendering. 
    \item Melakukan pengujian performa untuk menganalisis efisiensi dan pemanfaatan sumber daya \textit{framework} yang dikembangkan (memori, latensi input, dan waktu render). 
\end{enumerate}
\par

\section{Batasan Masalah} \label{I.Batasan}
Untuk menjaga ruang lingkup penelitian agar tetap terfokus dan realistis, penelitian ini dibatasi oleh hal-hal berikut: \par
\begin{enumerate}[noitemsep]
    \item \textit{Framework} yang dikembangkan berfokus pada arsitektur dasar \textit{desktop environment}, meliputi manajemen jendela/surface, input, dan rendering, tanpa implementasi penuh fitur desktop seperti panel, \textit{launcher}, atau \textit{file manager}. 
    \item Penelitian ini hanya menggunakan Wayland sebagai \textit{backend} tampilan utama, tanpa dukungan terhadap X11 atau kompositor lain. 
    \item Bahasa pemrograman yang digunakan adalah C++ dengan memanfaatkan pustaka Wayland (\textit{libwayland-server}) dan pustaka grafis OpenGL/Vulkan secara langsung. 
    \item Pengujian performa difokuskan pada efisiensi memori, waktu render, dan tingkat latensi input, bukan pada fitur antarmuka pengguna atau pengalaman pengguna secara keseluruhan. 
\end{enumerate}
\par
Framework ini dapat dikembangkan lebih lanjut dengan integrasi modul \textit{XWayland} untuk kompatibilitas aplikasi berbasis X11, sebagai pengembangan lanjutan di luar lingkup penelitian ini. \par

\section{Manfaat Penelitian} \label{I.Manfaat}
Penelitian ini diharapkan memberikan beberapa manfaat sebagai berikut: \par
\begin{enumerate}[noitemsep]
    \item \textbf{Bagi mahasiswa:} menambah pemahaman tentang penerapan arsitektur \textit{Data-Oriented Design} dan \textit{Entity-Component-System} dalam pengembangan sistem tingkat rendah. 
    \item \textbf{Bagi program studi Teknik Informatika:} menyediakan referensi penelitian di bidang sistem operasi, grafika komputer, dan arsitektur perangkat lunak berbasis DOD/ECS. 
    \item \textbf{Bagi dunia akademik dan penelitian:} memberikan kontribusi terhadap kajian pengembangan \textit{framework} grafis dan \textit{desktop environment} yang lebih efisien dan modular. 
    \item \textbf{Bagi komunitas pengembang \textit{open-source}:} menyediakan landasan arsitektur alternatif yang dapat dikembangkan menjadi kompositor atau \textit{window manager} ringan berbasis Wayland. 
\end{enumerate}
\par

\section{Sistematika Penulisan} \label{I.Sistematika}
\subsection*{Bab I}
Bab ini berisikan penjelasan latar belakang penelitian, rumusan masalah, tujuan, batasan, manfaat penelitian, serta sistematika penulisan tugas akhir. \par

\subsection*{Bab II}
Bab ini membahas tinjauan pustaka dan dasar teori, termasuk konsep Wayland, kompositor, paradigma \textit{Entity-Component-System} (ECS) dan \textit{Data-Oriented Design} (DOD), serta penelitian terdahulu yang relevan. \par

\subsection*{Bab III}
Bab ini menjelaskan metodologi penelitian, perancangan sistem, perangkat lunak dan pustaka yang digunakan, serta rancangan pengujian performa dan efisiensi \textit{framework}. \par

\subsection*{Bab IV}
Bab ini menyajikan hasil implementasi \textit{framework}, pengujian performa, analisis hasil, serta pembahasan mengenai efisiensi dan modularitas sistem yang dikembangkan. \par

\subsection*{Bab V}
Bab ini berisi kesimpulan dari penelitian yang dilakukan serta saran untuk pengembangan lebih lanjut, seperti ekspansi fitur, optimalisasi rendering, dan integrasi dengan komponen \textit{desktop environment} lainnya. \par



    %\input{chapters/chapter-2}
    %\input{chapters/chapter-3}
    %\input{chapters/chapter-4}
    %\input{chapters/chapter-5}
	%----------------------------------------------------------------%
    % Daftar Pustaka
    \newpage
    \phantomsection% 
    \addcontentsline{toc}{chapter}{DAFTAR PUSTAKA}
    \printbibliography[title={Daftar Pustaka}]
	%----------------------------------------------------------------%
    % Lampiran
    %----------------------------------------------------------------%
    % TODO: Tabel Lampiran
    \newpage
    \appendix
    \addcontentsline{toc}{chapter}{LAMPIRAN}
    \chapter*{Lampiran}
    \renewcommand\thesection{\Alph{section}}
    \input{chapters/appendix}
    %----------------------------------------------------------------%
\end{document}

% !TeX root = thesis.tex
%--------------------------------------------------------------------%
%
% Template TA LaTeX Teknik Informatika ITERA.
% Editor: Radhinka Bagaskara, Martin C.T. Manullang, I Wayan Wiprayoga Wisesa, Ardoni Yeriko Rifana Gultom (IF 2021)
% Version 2025.1
% TELAH DISESUAIKAN DENGAN FORMAT PERPUSTAKAAN ITERA MEI 2025 (UNESCO FORMAT)
%
% Berdasarkan "Templat LaTeX Tesis Informatika ITB" oleh Petra Barus & Peb Ruswono Aryan
% https://github.com/petrabarus/if-itb-latex
%--------------------------------------------------------------------%
%
% Berkas ini berisi struktur utama dokumen LaTeX yang akan dibuat.
%
%--------------------------------------------------------------------%

% Set jenis dokumen Tugas Akhir
\documentclass[article]{report} % Untuk versi perpustakaan
\usepackage[paperwidth=155mm,paperheight=230mm,top=2cm, bottom=2cm, left=2cm, right=2cm, includefoot, heightrounded, bindingoffset=0.5cm]{geometry}

%-------------------------------------------------------------------%
%
% Konfigurasi dokumen LaTeX untuk laporan tesis IF ITB
% 
%
% @author Radhinka Bagaskara, Martin C.T. Manullang, Petra Novandi (ITB)
%
%-------------------------------------------------------------------%
%
% Berkas asli berasal dari Steven Lolong
%
%-------------------------------------------------------------------%

% Import package penting
\usepackage[utf8]{inputenc}
\usepackage{subcaption} % Paket untuk mengatur gambar berdampingan
\usepackage{graphicx}
\usepackage{titling}
\usepackage{blindtext} % Untuk lorem ipsum
\usepackage{sectsty} % Untuk header & judul
\usepackage{chngcntr} % Untuk penambahan nomor caption
\usepackage{etoolbox} % Untuk CRUD variabel (?)
\usepackage{array} % % Untuk tabel di math mode
\usepackage{float} % Untuk tabular
\usepackage{longtable} % Untuk tabel yang potong halaman
\usepackage{amsmath} % Untuk equation
\usepackage{enumitem} % Untuk list enumerate yg lebih rapi
\usepackage[bookmarks]{hyperref}
\hypersetup{
	colorlinks,
	citecolor=black,
	filecolor=black,
	linkcolor=black,
	urlcolor=black
}

% Ukuran kertas A4
\special{papersize=210mm,297mm}

% Setting margin
\usepackage[top=3cm,bottom=3cm,left=3.5cm,right=3cm]{geometry}

% Setting indensasi (untuk halaman-halaman awal sebelum Bab I-V)
\usepackage{indentfirst}
\usepackage{parskip}
\setlength{\parindent}{20pt}
\setlength{\parskip}{10pt}

% Linespacing 1.5. Tidak serupa dengan 1.5 di Word (RDB)
\renewcommand{\baselinestretch}{1.5}

% Agar tidak ada kata yang terpotong setiap baris kalimat
\hyphenpenalty=10000

% Font
%\usepackage{mathptmx} 
\usepackage{newtx} 
% Times New Roman itu copyright dari Microsoft. Ini alternatifnya (RDB)

% Judul bahasa Indonesia
\usepackage[bahasa]{babel}

% Format tanggal
\usepackage[style=ddmmyyyy,datesep=-]{datetime2}

% Format citation
\usepackage[backend=bibtex,citestyle=ieee]{biblatex} % Untuk bisa jalan di Texstudio, harus backend=bibtex

% Format Daftar Pustaka agar lebih rapi
% Remove "In:" before journal titles
\renewbibmacro{in:}{}
% Ensure URLs, DOIs, and ISSNs use the default font (e.g., Times New Roman)
\renewcommand*{\UrlFont}{\rmfamily} % Use default font for URLs
\DeclareFieldFormat{issn}{#1}       % Use default font for ISSN
\DeclareFieldFormat{doi}{#1}        % Use default font for DOI
% Remove DOI and URL fields if not needed (optional)
\AtEveryBibitem{
	\clearfield{doi}
	\clearfield{url}
}
\DeclareLanguageMapping{bahasa}{english}

% Package untuk link di daftar isi.
\usepackage{titlesec}       % Package Format judul
\usepackage{ragged2e}		% Alignment
\usepackage{multirow}		% Untuk bisa merge cell di tabel
\usepackage{tikz}			% Untuk menggambar kotak pas foto
\usepackage{setspace}		% Spacing paragraph
\usepackage{fancyhdr}		% Agar nomor halaman di pojok kanan atas
\usepackage[figurewithin=none, tablewithin=none]{caption} 		% Caption gambar & tabel
% Item di Daftar tabel, gambar, dan rumus harusnya tidak ada spasi jika
% berada di bab yg berbeda2
\usepackage[titles]{tocloft}% Tipografi untuk halaman Daftar. judul Daftar jadi di tengah & fontnya kecil

% Setting supaya nomor halaman pertama dengan "chapter"
% berada di kanan atas
\fancypagestyle{plain}{%
	\fancyhf{}%
	\renewcommand{\headrulewidth}{0pt}
	\fancyhead[R]{\thepage}
}

% Setting posisi & spasi judul
\chapterfont{\centering \large}
\titleformat{\chapter}[display]%
{\large\centering\bfseries}%
{\chaptertitlename\ \thechapter}{-3pt}% % Setting spasi antara "Bab" & "Judul Bab"
{\large\bfseries\uppercase}
\titleformat{\section}%
{\normalfont\normalsize\bfseries}{\thesection}{1em}{}
\titleformat{\subsection}%
{\normalfont\normalsize\bfseries}{\thesubsection}{1em}{}

% Setting spacing di setiap judul chapter
\titlespacing*{\chapter}{0pt}{-30pt}{20pt}

% Setting nomor pada subbsubsubbab
\setcounter{secnumdepth}{3}

% Counter untuk figure dan table, agar bertambah walaupun lintas subbab
\counterwithout{figure}{section}
\counterwithout{table}{section}

% Supaya tidak ada garis di header
\renewcommand{\headrulewidth}{0pt}

% Setting penomoran caption gambar, tabel, dan rumus
\renewcommand{\thefigure}{\arabic{chapter}.\arabic{figure}}
\renewcommand{\thetable}{\arabic{chapter}.\arabic{table}}
\renewcommand\theequation{\arabic{chapter}.\arabic{equation}}

% Setting spasi list di daftar isi, daftar gambar, daftar tabel, daftar rumus
\setlength{\cftbeforechapskip}{5.2pt}
\cftsetindents{section}{1.5em}{2.3em}
\cftsetindents{subsection}{3em}{3em}
\setlength{\cfttabindent}{1.5em}
\setlength{\cftfigindent}{1.5em}

% Mengkapitalkan judul Daftar Isi, Gambar, & Tabel
\addto\captionsbahasa{%
	\renewcommand{\contentsname}{DAFTAR ISI}%
	\renewcommand{\listfigurename}{DAFTAR GAMBAR}%
	\renewcommand{\listtablename}{DAFTAR TABEL}%
}

% Tambahkan kata "BAB" sebelum nomor bab di daftar isi
\renewcommand*\cftchappresnum{\MakeUppercase{BAB}~}
\renewcommand\chaptername{BAB}
\settowidth{\cftchapnumwidth}{\cftchappresnum}
\renewcommand{\cftchapaftersnumb}{\quad}
\addtocontents{toc}{
	\protect\renewcommand*\protect\cftchappresnum{\MakeUppercase{\chaptername}~}
}

% Menambahkan titik2 antara judul bab & halaman di halaman Daftar
\renewcommand{\cftchapleader}{\dotfill} 
\renewcommand{\cftsecleader}{\dotfill}
\renewcommand{\cftsubsecleader}{\dotfill}
\renewcommand{\cftfigleader}{\dotfill}
\renewcommand{\cfttableader}{\dotfill}

% Setting Daftar Rumus
\newcommand{\listequationsname}{DAFTAR RUMUS}
\newlistof{myequations}{equ}{\listequationsname}
\newcommand{\myequations}[1]{
	\addcontentsline{equ}{myequations}{\protect\numberline{\theequation}#1}
}
\renewcommand{\cftmyequationspresnum}{Rumus\ } % Agar list rumus di Daftar Rumus ada tulisan "Rumus x.y"
\newlength{\mylenf}
\settowidth{\mylenf}{\cftmyequationspresnum}
\setlength{\cftmyequationsnumwidth}{\dimexpr\mylenf+1.5em} % Menyesuaikan nomor
\setlength{\cftmyequationsindent}{1.5em} % Menambahkan indentasi daftar rumus
\renewcommand{\cftmyequationsleader}{\dotfill}

% Setting agar list item di Daftar Gambar & Tabel ada tulisan "Gambar/Tabel x.y"
\renewcommand\cftfigpresnum{Gambar\ }
\renewcommand\cfttabpresnum{Tabel\ }
\settowidth{\mylenf}{\cftfigpresnum}
\setlength{\cftfignumwidth}{\dimexpr\mylenf+1.5em}
\setlength{\cfttabnumwidth}{\dimexpr\mylenf+0.5em}

% Setting judul Daftar menjadi di tengah & ukuran large
%\renewcommand{\cfttoctitlefont}{\hspace*{\fill}\large\bfseries}
%\renewcommand{\cftaftertoctitle}{\hspace*{\fill}}
%\renewcommand{\cftlottitlefont}{\hspace*{\fill}\large\bfseries}
%\renewcommand{\cftafterlottitle}{\hspace*{\fill}}
%\renewcommand{\cftloftitlefont}{\hspace*{\fill}\large\bfseries}
%\renewcommand{\cftafterloftitle}{\hspace*{\fill}}
%%\renewcommand{\cftmyequationstitlefont}{\hspace*{\fill}\large\bfseries}
%%\renewcommand{\cftaftertoctitle}{\hspace*{\fill}}

% Untuk mengganti nama bulan di babel bahasa
% tapi belum jalan (RDB)
\StartBabelCommands*{bahasa}{date}
\SetStringLoop{month#1name}{%
	Januari,Februari,Maret,April,Mei,Juni,%
	Juli,Agustus,September,Oktober,November,%
	Desember}
\EndBabelCommands     

% english title
\providecommand\titleEN[1]{\providecommand\thetitleEN{#1}}

% Saya lupa ini buat apa (RDB)
%\renewcommand{\theHsection}{\thepart.section.\thesection}

% Semua dibawah ini berhubungan dengan CRUD variabel (ada simbol @)
\makeatletter % Jangan dihapus

% Command untuk variabel NIM
\newcommand{\nim}[1]{\def\@nim{#1}}
\newcommand{\printnim}{\@nim}

% Command untuk variabel Dosen Pembimbing I & II
\newcommand{\namadosbinga}[1]{\def\@namadosbinga{#1}}
\newcommand{\namadosbingb}[1]{\def\@namadosbingb{#1}}
\newcommand{\nipdosbinga}[1]{\def\@nipdosbinga{#1}}
\newcommand{\nipdosbingb}[1]{\def\@nipdosbingb{#1}}
\newcommand{\printnamadosbinga}{\@namadosbinga}
\newcommand{\printnamadosbingb}{\@namadosbingb}
\newcommand{\printnipdosbinga}{\@nipdosbinga}
\newcommand{\printnipdosbingb}{\@nipdosbingb}
\newcommand{\dosbingA}[2]{\namadosbinga{#1} \nipdosbinga{#2}}
\newcommand{\dosbingB}[2]{\namadosbingb{#1} \nipdosbingb{#2}}

% Command untuk variabel Dosen Penguji I & II
\newcommand{\namapengujia}[1]{\def\@namapengujia{#1}}
\newcommand{\namapengujib}[1]{\def\@namapengujib{#1}}
\newcommand{\nippengujia}[1]{\def\@nippengujia{#1}}
\newcommand{\nippengujib}[1]{\def\@nippengujib{#1}}
\newcommand{\printnamapengujia}{\@namapengujia}
\newcommand{\printnamapengujib}{\@namapengujib}
\newcommand{\printnippengujia}{\@nippengujia}
\newcommand{\printnippengujib}{\@nippengujib}
\newcommand{\pengujiA}[2]{\namapengujia{#1} \nippengujia{#2}}
\newcommand{\pengujiB}[2]{\namapengujib{#1} \nippengujib{#2}}

% Command untuk merubah spacing equation
\g@addto@macro\normalsize{%
	\setlength\abovedisplayskip{-10pt}
	\setlength\belowdisplayskip{-10pt}
	\setlength\abovedisplayshortskip{-10pt}
	\setlength\belowdisplayshortskip{-10pt}
}

\makeatother % Jangan dihapus

\bibliography{references}

\begin{document}
% 	\pagestyle{plain}
% 	\fancyhf{}
% 	\rfoot{Halaman \thepage}%

    %----------------------------------------------------------------%
    % Konfigurasi Informasi Tugas Akhir
    %----------------------------------------------------------------%
    
    % Judul Tugas Akhir
    \title{Analisis Algoritma ABC Untuk Pemecahan Masalah Penjadwalan Job Shop} % Judul Tugas Akhir dalam Bahasa Indonesia	
    % DITULIS DALAM HURUF KAPITAL; Font size 16 pt; Bold; Tidak melebihi 4 baris
    \titleEN{Comparison }      % Judul Tugas Akhir dalam Bahasa Inggris
    
    % Informasi Mahasiswa
    \author{Ardoni Yeriko Rifana Gultom}		% Nama Mahasiswa
	\nim{121140141}			% NIM Mahasiswa
	
	%Informasi Dosen Pembimbing
	\dosbingA%
		{Martin Clinton Tosima Manullang, Ph.D.}%	% Nama Dosen Pembimbing 1
		{NIP. 19930109 2019 03 1 017}				% NIP Dosen Pembimbing 1
	\dosbingB%
		{Nama dan Gelar Pembimbing II}%	% Nama Dosen Pembimbing 2
		{NIP. 123456789}				% NIP Dosen Pembimbing 2
		
	%Informasi Dosen Penguji
	\pengujiA%
		{Andika Setiawan, S.Kom., M.Cs.}%	% Nama Dosen Penguji 1
		{NIP. 19911127 2022 03 1 007}				% NIP Dosen Penguji 1
	\pengujiB%
		{Eko Dwi Nugroho, S.Kom., M.Cs.}%	% Nama Dosen Penguji 2
		{NIP. 19910209 2024 06 1 001}				% NIP Dosen Penguji 2

	\sloppy % mencegah text overflow. (Jose)
    \pagenumbering{roman}
    \setcounter{page}{1} % Nomor halaman dimulai dengan "ii" di hal. Pengesahan

    \clearpage
\pagestyle{empty}

\begin{center}
	\smallskip
	
	\begin{center}
		\fontsize{11pt}{11pt}
		\bfseries \MakeUppercase{\thetitle}
		\vfill
	    \uppercase{Tugas Akhir}
	    \vfill
		\normalfont Diajukan sebagai syarat menyelesaikan jenjang strata Satu (S-1) di Program Studi Teknik Informatika, Fakultas Teknologi Industri, Institut Teknologi Sumatera
		\vfill
	\end{center}

	\large \bfseries Oleh:\\
    \large \bfseries \theauthor\\
    \printnim
    \vfill
    
    \begin{figure}[h]
    	\centering
    	\includegraphics[width=3cm, keepaspectratio]{figure/Logo_ITERA.png}
    \end{figure}
    \vfill

	\begin{center}
		\fontsize{11pt}{11pt}
	    \bfseries
	    \uppercase{
	        Program Studi Teknik Informatika \\
	        Fakultas Teknologi Industri\\
	        Institut Teknologi Sumatera\\
	        Lampung Selatan
	    }\\
	    \the\year{}
    \end{center}

\end{center}

\clearpage
 % Hardcover
   \clearpage
\pagestyle{fancy}
\fancyhf{}
\fancyhead[R]{\thepage}
\phantomsection% 
\addcontentsline{toc}{chapter}{LEMBAR PENGESAHAN}

\begin{center}

	\large \bfseries \MakeUppercase{Lembar Pengesahan}
    
    \small \normalfont \singlespacing \justify{
    Saya menyatakan bahwa Tugas Akhir berjudul “{\thetitle}" merupakan hasil karya saya sendiri dan belum pernah diajukan, baik sebagian maupun seluruhnya, di Institut Teknologi Sumatera atau institusi pendidikan lain oleh saya maupun pihak lain.}
    %Tugas Akhir Sarjana dengan judul "{\thetitle}" adalah benar dibuat oleh saya sendiri dan belum pernah dibuat dan diserahkan sebelumnya, baik sebagian ataupun seluruhnya, baik oleh saya ataupun orang lain, baik di Institut Teknologi Sumatera maupun di institusi pendidikan lainnya.

	% Informasi Mahasiswa
    \flushleft
	\setlength{\tabcolsep}{0pt}
	\begin{tabular}{p{0.59\textwidth} p{0.3\textwidth}}
        \vspace{0.1cm}
		Lampung Selatan, \todayIndo & %
		\multirow{6}{*}{
			% Kotak pasfoto 3x4
			\phantom{----------------------} % Amazing hack biar kotaknya ke kanan (RDB)
			\begin{tikzpicture}
				\draw rectangle (2cm,3cm) node[pos=0.5] {Foto 2x3};
			\end{tikzpicture}
		}\\
		Penulis, \\
		& \\
		& \\
		%& \\
		\theauthor\\
		NIM. \printnim
	\end{tabular}
	% Informasi Dosen
	\vspace{0.4cm}
        \begin{center}
        Diperiksa dan disetujui oleh,
        \end{center}
        \vspace{0.1cm}

	\justify
    \setlength{\tabcolsep}{0pt}
    \begin{tabular}{ m{0.5cm}  m{0.7\textwidth} >{\centering\arraybackslash}m{0.3\textwidth}}
        \multicolumn{2}{c}{\hspace*{70pt}Pembimbing} & \multicolumn{1}{c}{} \\[2pt]
		1. & \printnamadosbinga & \\
		 & \printnipdosbinga & ............... \\%[4pt]
		 & \\
		2. & \printnamadosbinga & \\
		& \printnipdosbinga & ............... \\%[4pt]
		& \\
		\multicolumn{2}{c}{\hspace*{70pt}Penguji} & \multicolumn{1}{c}{} \\[2pt]
		1. & \printnamapengujia & \\
		& \printnippengujia & ............... \\[4pt]
            %& \\
		2. & \printnamapengujib & \\
		& \printnippengujib & ............... \\
    \end{tabular}
%	\vfill

	\begin{center}
		\fontsize{10pt}{10pt}
        \vspace{0.45cm}
		Disahkan oleh,\\
		Koordinator Program Studi Teknik Informatika\\
		Fakultas Teknologi Industri\\
		Institut Teknologi Sumatera
		\vspace{1.8cm}\\
		Andika Setiawan, S.Kom., M.Cs. \\ % TODO: make automatic
		NIP. 19911127 2022 03 1 007 \\
	\end{center}
	\vfill

\end{center}
\clearpage % Lembar Pengesahan
    % Ganti margin & aktifkan twoside mulai dari sini
    \clearpage
    % \newgeometry{
    %     top=2cm,
    %     bottom=2cm,
    %     left=2cm,
    %     right=2cm,
    %     includefoot,
    %     heightrounded,
    %     bindingoffset=0.5cm,
    %     twoside
    % }
    \pagestyle{alternatingstyle}
   \clearpage
\phantomsection% 
\addcontentsline{toc}{chapter}{Halaman Pernyataan Orisinalitas}

\begin{center}
	\smallskip
	
%	\chapter*{\normalsize{Halaman Pernyataan Orisinalitas}}
	\large \bfseries \MakeUppercase{Halaman Pernyataan Orisinalitas} \linebreak
	
	\normalsize \onehalfspacing{
		Tugas Akhir dengan judul “{\thetitle}” adalah karya saya sendiri, dan semua sumber baik yang dikutip maupun dirujuk telah saya nyatakan benar. }
	\vspace{3cm}
	
	\centering 
	\begin{tabular}{l l}
		Nama 			& : \theauthor \\
		& \\
		NIM 			& : \printnim \\
		& \\
		\\
		Tanda Tangan 	& : ................................... \\
		& \\
		Tanggal 		& : ................................... \\
	\end{tabular}
	
\end{center}
\clearpage
 % Halaman Pernyataan Orisinalitas
   \clearpage
\phantomsection% 
\addcontentsline{toc}{chapter}{Halaman Persetujuan Publikasi}

\begin{center}
	\smallskip
	
	\normalsize \bfseries \MakeUppercase{
		HALAMAN PERNYATAAN PERSETUJUAN PUBLIKASI \\
		TUGAS AKHIR UNTUK KEPENTINGAN AKADEMIS
	}\linebreak
	
	\normalsize \normalfont \onehalfspacing \justifying{
		Sebagai civitas akademik Institut Teknologi Sumatera, saya yang bertanda tangan di bawah ini:}
	
	\flushleft
	\setlength{\tabcolsep}{0pt}
	\begin{tabular}{l l}
		Nama 			&  : \\
		NIM 			&  : \\
		Program Studi \	&  : \\
		Jurusan 		&  : \\
		Jenis Karya 	&  : \\
	\end{tabular}

	\justifying
	demi pengembangan ilmu pengetahuan, menyetujui untuk memberikan kepada Institut Teknologi Sumatera Hak Bebas Royalti Noneksklusif (Non-exclusive Royalty Free Right) atas karya ilmiah saya yang berjudul: 
	
	\centering
	\thetitle
	
	\justifying
	beserta perangkat yang ada (jika diperlukan). Dengan Hak Bebas Royalti Noneksklusif ini Institut Teknologi Sumatera berhak menyimpan, mengalihmedia/formatkan, mengelola dalam bentuk pangkalan data (database), merawat, dan memublikasikan tugas akhir saya selama tetap mencantumkan nama saya sebagai penulis/pencipta dan sebagai pemilik Hak Cipta.
	
	Demikian pernyataan ini saya buat dengan sebenarnya. \\
	
	\centering
	Dibuat di : Lampung Selatan\\
	Pada tanggal : (tanggal bulan tahun)\\
	\vspace{3cm}
	Yang menyatakan (\theauthor)
	
	
\end{center}
\clearpage
 % Halaman Persetujuan Publikasi
   \clearpage
\phantomsection% 
\addcontentsline{toc}{chapter}{KATA PENGANTAR}
%\thispagestyle{fancy}

\begin{justifying}
	\large \bfseries \centering \MakeUppercase{Kata Pengantar}\linebreak
	
	\normalsize \normalfont \justifying
	\textit{Pada halaman ini mahasiswa berkesempatan untuk menyatakan terima kasih secara tertulis kepada pembimbing dan pihak lain yang telah memberi bimbingan, nasihat, saran dan kritik, kepada mereka yang telah membantu melakukan penelitian, kepada perorangan atau lembaga yang telah memberi bantuan keuangan, materi dan/atau sarana. Cara menulis kata pengantar beraneka ragam, tetapi hendaknya menggunakan kalimat yang baku. Ucapan terima kasih agar dibuat tidak berlebihan dan dibatasi pada pihak yang terkait secara ilmiah (berhubungan dengan subjek/materi penelitian). } \par
	Puji syukur kehadirat Allah SWT/Tuhan Yang Maha Esa atas limpahan rahmat, karunia, serta petunjuk-Nya sehingga penyusunan tugas akhir ini telah terselesaikan dengan baik. Dalam penyusunan tugas akhir ini penulis telah banyak mendapatkan arahan, bantuan, serta dukungan dari berbagai pihak. Oleh karena itu pada kesempatan ini penulis mengucapan terima kasih kepada: \par
	\begin{enumerate}
		\item {[Rektor ITERA]} selaku Rektor Institut Teknologi Sumatera.  
		\item {[Dekan FTI]} selaku Dekan Fakultas Teknologi Industri.
		\item {[Koor Prodi IF]} selaku Ketua Program Studi Teknik Informatika.
		\item {[Dosen Pembimbing]} selaku Dosen Pembimbing atas ide, waktu, tenaga, perhatian, dan masukan yang telah disumbangsihkan kepada penulis.
		\item {[Isi nama lainnya]}
	\end{enumerate} \par
	Akhir kata penulis berharap semoga tugas akhir ini dapat memberikan manfaat bagi kita semua.
	\vfill
	
\end{justifying}
\clearpage





\begin{comment}
\clearpage
\pagestyle{fancy}
\fancyhf{}
\fancyhead[R]{\thepage}
\phantomsection% 
%\clearpage
%\phantomsection% 
\addcontentsline{toc}{chapter}{KATA PENGANTAR}
%\thispagestyle{fancy}

\begin{justifying}
	\large \bfseries \centering \MakeUppercase{Kata Pengantar}\linebreak
	
	\normalsize \normalfont \justifying
	Puji syukur kehadirat Tuhan Yang Maha Esa atas limpahan rahmat, karunia, serta petunjuk-Nya sehingga penyusunan tugas akhir ini telah terselesaikan dengan baik. Dalam penyusunan tugas akhir ini penulis telah banyak mendapatkan arahan, bantuan, serta dukungan dari berbagai pihak. Oleh karena itu pada kesempatan ini penulis mengucapan terima kasih kepada: \par
	\begin{enumerate}
		\item Bapak Prof. Dr. I. Nyoman Pugeg Aryantha selaku Rektor Institut Teknologi Sumatera.  
		\item Bapak Hadi Teguh Yudistira, S.T., Ph.D. selaku Dekan Fakultas Teknologi Industri.
		\item Bapak Andika Setiawan, S. Kom., M. Cs. selaku Ketua Program Studi Teknik Informatika.
		\item Bapak Ilham Firman Ashari, S. Kom., M.T. selaku Koordinator Tugas Akhir Program Studi Teknik Informatika.  
		\item Bapak Martin C. T. Manullang, Ph.D. selaku Dosen Pembimbing atas ide, waktu, tenaga, perhatian, dan masukan yang telah disumbangsihkan kepada penulis.
            \item Bapak Andika Setiawan, S. Kom., M. Cs dan Bapak Eko Dwi Nugroho, S.Kom., M.Cs. selaku Dosen Penguji atas saran dan masukan yang diberikan. 
		\item Kedua Orang Tua dan Adik yang selalu memberikan dukungan dan doa sehingga penulis dapat menyelesaikan tugas akhir ini. Kelembutan, dukungan, dan cinta yang kalian berikan selalu menjadi sumber inspirasi dan kekuatan.
		\item Teman-teman penulis yang membantu selama masa perkuliahan yang tidak bisa disebutkan satu persatu.
	\end{enumerate} \par
	Akhir kata penulis berharap semoga tugas akhir ini dapat memberikan manfaat bagi kita semua. Penulis menyadari bahwa tugas akhir ini tidak luput dari kekurangan dan kelemahan, dan penulis  terbuka untuk menerima saran, kritik, dan masukan yang membangun.
	\vfill
	
\end{justifying}
\clearpage
\end{comment} % Kata Pengantar
    
   \clearpage
\phantomsection% 
\addcontentsline{toc}{chapter}{RINGKASAN}
\thispagestyle{fancy}

\begin{center}
	\large \bfseries \MakeUppercase{Ringkasan}\\
	\normalsize \normalfont {\thetitle}\\
	\normalsize \normalfont {\theauthor}\\
	\bigskip
	
	\normalsize \normalfont \justifying
	\normalsize \normalfont \justifying
	\textit{Halaman Ringkasan berisi uraian singkat tentang latar belakang masalah, rumusan masalah, tujuan, metodologi penelitian, hasil dan analisis data, serta kesimpulan dan saran. Isi ringkasan tidak lebih dari 1000 kata (sekitar maksimal 2 halaman).}\par
	\lipsum[1-2] % Menampilkan paragraf 1 sampai 2 dari lorem ipsum

	\vfill
	
\end{center}
\clearpage % Ringkasan
    \clearpage

\singlespacing{
	\textbf{\thetitle}\\
	\mbox{\theauthor \ (\printnim)}\\
	Pembimbing I \printnamadosbinga\\
	Pembimbing II \printnamadosbingb\\
}

%\chapter*{ABSTRAK}
\normalsize \bfseries \centering \MakeUppercase{Abstrak}
\phantomsection% 
\addcontentsline{toc}{chapter}{Abstrak}
\\[2\baselineskip]

%taruh abstrak bahasa indonesia di sini
\justifying \normalfont \normalsize{
	\blindtext
}

\textbf{Kata Kunci}: Kata Kunci 1, Kata Kunci 2
\clearpage % Abstrak (Indonesia)
    \clearpage

\singlespacing{
	\textbf{English Title}\\
	\mbox{\theauthor \ (\printnim)}\\
	Pembimbing I \printnamadosbinga\\
	Pembimbing II \printnamadosbingb\\
}

%\chapter*{ABSTRAK}
\normalsize \bfseries \centering \MakeUppercase{Abstract}
\phantomsection% 
\addcontentsline{toc}{chapter}{Abstract}
\\[2\baselineskip]

%taruh abstrak bahasa inggris di sini
\justifying \normalfont \normalsize{
	\blindtext
}

Keyword: Keyword 1, Keyword 2
\clearpage % Abstrak (Inggris)
%    \clearpage

\normalsize \bfseries \centering \MakeUppercase{Motto}
\phantomsection% 
\addcontentsline{toc}{chapter}{Motto}
\\[2\baselineskip]

\justifying \normalfont{
	% Motto
	\blindtext
}

\clearpage
%   \clearpage

% PS: Ada bug dimana jika menge-build dari file ini, ada error. Tapi halamannya
% sendiri tidak error jika dibuild dari file lain. (Radhinka)

\normalsize \bfseries \centering \MakeUppercase{Persembahan}
\phantomsection% 
\addcontentsline{toc}{chapter}{Persembahan}
\\[2\baselineskip]

\justifying \normalfont{
	% Kata-kata persembahan
	\blindtext
}

\clearpage

    % Daftar Isi
    \phantomsection% 
    \addcontentsline{toc}{chapter}{DAFTAR ISI}
    \tableofcontents
    \pagebreak
    % Daftar tabel
    \phantomsection% 
    \addcontentsline{toc}{chapter}{DAFTAR TABEL}
    \listoftables
    \pagebreak
    % Daftar gambar
    \phantomsection% 
    \addcontentsline{toc}{chapter}{DAFTAR GAMBAR}
    \listoffigures
    \pagebreak
    % Daftar rumus
    % \pagestyle{daftarrumusstyle}
    \phantomsection% 
    \addcontentsline{toc}{chapter}{DAFTAR RUMUS}
    \listofmyequations
    % \thispagestyle{daftarrumusstyle}
    % \pagebreak
%    \clearpage

\chapter*{Daftar Simbol}
\thispagestyle{fancy}
\fancyhf{}
\fancyhead[R]{\thepage}

\justifying
(Tuliskan maksud penulisan laporan, misal “Laporan penelitian ini dimaksud kan untuk memenuhi salah ”.........Pada halaman ini mahasiswa berkesempatan untuk menyatakan terima kasih secara tertulis kepada pembimbing dan pihak lain yang telah memberi bimbingan, nasihat, saran dan kritik, kepada mereka yang telah membantu melakukan penelitian, kepada perorangan atau lembaga yang telah memberi bantuan keuangan, materi dan/atau sarana.

Cara menulis kata pengantar beraneka ragam, tetapi hendaknya menggunakan kalimat yang baku. Ucapan terima kasih agar dibuat tidak berlebihan dan dibatasi pada pihak yang terkait secara ilmiah (berhubungan dengan subjek/materi penelitian). 

    % Daftar kode
    \phantomsection% 
    \addcontentsline{toc}{chapter}{DAFTAR KODE}
    \lstlistoflistings
    \pagebreak
    

    %----------------------------------------------------------------%
    % Konfigurasi Bab
    %----------------------------------------------------------------%
    \renewcommand{\chaptername}{BAB}
    % Bab: Arabic
    \renewcommand{\thechapter}{\Roman{chapter}}
    % Sub-bab: Roman
    \renewcommand\thesection{\arabic{chapter}.\arabic{section}}
    
    % Setting supaya nomor halaman pertama dengan "chapter"
    % berada di tengah bawah, tapi selanjut2nya di kanan atas
    \fancypagestyle{plain}{%
    	\fancyhf{}%
    	\renewcommand{\headrulewidth}{0pt}
    	\fancyhead[]{}
    	\fancyfoot[C]{\thepage}
    }
    % Reset penomoran halaman menjadi 1
    \setcounter{page}{1}
    \pagenumbering{arabic}
    
    %----------------------------------------------------------------%
    %----------------------------------------------------------------%
    % Daftar Bab
    % Untuk menambahkan daftar bab, buat berkas bab misalnya `chapter-6` di direktori `chapters`, dan masukkan ke sini.
    %----------------------------------------------------------------%
    \justifying
    \newpage
\chapter{PENDAHULUAN} \label{Bab I}

\section{Latar Belakang} \label{I.Latar Belakang}
Perkembangan sistem operasi modern, khususnya berbasis Linux, mendorong kebutuhan akan lingkungan desktop (\textit{desktop environment}) yang lebih efisien, modular, dan dapat beradaptasi dengan teknologi grafis baru seperti Wayland. Wayland merupakan protokol komunikasi antara \textit{compositor} dan aplikasi klien yang menggantikan X11 sebagai standar tampilan modern. Dengan desain yang sederhana, efisien, dan berbasis objek pada tataran protokol, Wayland memberi ruang bagi pengembang untuk membangun sistem tampilan dengan latensi rendah dan arsitektur modular. \par

Dalam praktiknya, ekosistem \textit{desktop environment} arus utama masih bertumpu pada kerangka berparadigma objek atau modular tradisional. GNOME (Mutter/GTK) menggunakan \textit{GObject} sebagai sistem objek di C, KDE Plasma/KWin berbasis Qt (QObject dan \textit{signals/slots}), sedangkan di ranah kompositor Wayland, Sway memanfaatkan \textit{wlroots} (pustaka modular di C). Hyprland ditulis dalam C++ mengintegrasikan Aquamarine sebagai \textit{rendering backend}. Pendekatan tersebut efektif dan matang, namun umumnya belum mengadopsi arsitektur \textit{Entity-Component-System} (ECS). \par

Di sisi lain, paradigma \textit{Data-Oriented Design} (DOD) yang diimplementasikan melalui arsitektur ECS menawarkan penataan data yang lebih bersahabat terhadap \textit{cache} dan paralelisasi, sehingga berpotensi meningkatkan efisiensi jalur eksekusi kritis (misalnya manajemen jendela, komposisi frame, dan penanganan input). Pendekatan ini telah terbukti efektif dalam dunia \textit{game engine} (misal Unity/Bevy), tetapi penerapannya pada sistem \textit{desktop environment} dan kompositor Wayland masih terbatas. \par

Wayland sebagai protokol tidak mengikat arsitektur internal kompositor; karena itu, terdapat peluang penelitian untuk merancang dan mengimplementasikan sebuah \textit{framework} \textit{desktop environment} berbasis ECS di atas Wayland. Framework semacam ini ditujukan untuk mengelola entitas seperti jendela/surface, perangkat input, dan objek render secara efisien, sekaligus menjaga modularitas agar mudah diperluas. Dengan demikian, penelitian ini memposisikan ECS/DOD bukan sebagai klaim pengganti OOP secara umum, melainkan sebagai eksplorasi arsitektur alternatif pada domain \textit{desktop environment} modern. \par

\section{Rumusan Masalah} \label{I.Rumusan Masalah}
Berdasarkan latar belakang tersebut, maka rumusan masalah dalam penelitian ini adalah sebagai berikut: \par
\begin{enumerate}[noitemsep]
    \item Bagaimana merancang arsitektur \textit{framework} \textit{desktop environment} berbasis \textit{Entity-Component-System} (ECS) yang dapat beroperasi di atas protokol Wayland? 
    \item Bagaimana mengimplementasikan \textit{framework} tersebut sehingga mampu mengelola entitas sistem seperti jendela/surface, input, dan rendering secara efisien? 
    \item Bagaimana mengevaluasi performa dan efisiensi \textit{framework} yang dirancang terhadap pemanfaatan sumber daya sistem (memori, latensi input, dan waktu render)? 
\end{enumerate}
\par

\section{Tujuan Penelitian} \label{I.Tujuan}
Tujuan dari penelitian ini adalah sebagai berikut: \par
\begin{enumerate}[noitemsep]
    \item Merancang arsitektur \textit{framework} \textit{desktop environment} berbasis \textit{Entity-Component-System} (ECS) di atas protokol Wayland. 
    \item Mengimplementasikan prototipe \textit{framework} yang mendukung pengelolaan entitas utama seperti jendela/surface, perangkat input, dan sistem rendering. 
    \item Melakukan pengujian performa untuk menganalisis efisiensi dan pemanfaatan sumber daya \textit{framework} yang dikembangkan (memori, latensi input, dan waktu render). 
\end{enumerate}
\par

\section{Batasan Masalah} \label{I.Batasan}
Untuk menjaga ruang lingkup penelitian agar tetap terfokus dan realistis, penelitian ini dibatasi oleh hal-hal berikut: \par
\begin{enumerate}[noitemsep]
    \item \textit{Framework} yang dikembangkan berfokus pada arsitektur dasar \textit{desktop environment}, meliputi manajemen jendela/surface, input, dan rendering, tanpa implementasi penuh fitur desktop seperti panel, \textit{launcher}, atau \textit{file manager}. 
    \item Penelitian ini hanya menggunakan Wayland sebagai \textit{backend} tampilan utama, tanpa dukungan terhadap X11 atau kompositor lain. 
    \item Bahasa pemrograman yang digunakan adalah C++ dengan memanfaatkan pustaka Wayland (\textit{libwayland-server}) dan pustaka grafis OpenGL/Vulkan secara langsung. 
    \item Pengujian performa difokuskan pada efisiensi memori, waktu render, dan tingkat latensi input, bukan pada fitur antarmuka pengguna atau pengalaman pengguna secara keseluruhan. 
\end{enumerate}
\par
Framework ini dapat dikembangkan lebih lanjut dengan integrasi modul \textit{XWayland} untuk kompatibilitas aplikasi berbasis X11, sebagai pengembangan lanjutan di luar lingkup penelitian ini. \par

\section{Manfaat Penelitian} \label{I.Manfaat}
Penelitian ini diharapkan memberikan beberapa manfaat sebagai berikut: \par
\begin{enumerate}[noitemsep]
    \item \textbf{Bagi mahasiswa:} menambah pemahaman tentang penerapan arsitektur \textit{Data-Oriented Design} dan \textit{Entity-Component-System} dalam pengembangan sistem tingkat rendah. 
    \item \textbf{Bagi program studi Teknik Informatika:} menyediakan referensi penelitian di bidang sistem operasi, grafika komputer, dan arsitektur perangkat lunak berbasis DOD/ECS. 
    \item \textbf{Bagi dunia akademik dan penelitian:} memberikan kontribusi terhadap kajian pengembangan \textit{framework} grafis dan \textit{desktop environment} yang lebih efisien dan modular. 
    \item \textbf{Bagi komunitas pengembang \textit{open-source}:} menyediakan landasan arsitektur alternatif yang dapat dikembangkan menjadi kompositor atau \textit{window manager} ringan berbasis Wayland. 
\end{enumerate}
\par

\section{Sistematika Penulisan} \label{I.Sistematika}
\subsection*{Bab I}
Bab ini berisikan penjelasan latar belakang penelitian, rumusan masalah, tujuan, batasan, manfaat penelitian, serta sistematika penulisan tugas akhir. \par

\subsection*{Bab II}
Bab ini membahas tinjauan pustaka dan dasar teori, termasuk konsep Wayland, kompositor, paradigma \textit{Entity-Component-System} (ECS) dan \textit{Data-Oriented Design} (DOD), serta penelitian terdahulu yang relevan. \par

\subsection*{Bab III}
Bab ini menjelaskan metodologi penelitian, perancangan sistem, perangkat lunak dan pustaka yang digunakan, serta rancangan pengujian performa dan efisiensi \textit{framework}. \par

\subsection*{Bab IV}
Bab ini menyajikan hasil implementasi \textit{framework}, pengujian performa, analisis hasil, serta pembahasan mengenai efisiensi dan modularitas sistem yang dikembangkan. \par

\subsection*{Bab V}
Bab ini berisi kesimpulan dari penelitian yang dilakukan serta saran untuk pengembangan lebih lanjut, seperti ekspansi fitur, optimalisasi rendering, dan integrasi dengan komponen \textit{desktop environment} lainnya. \par



    % Mulai selang-seling halaman
    \forceoddpage
    \cleardoublepage
    \pagestyle{alternatingstyle}
    \newpage
\chapter{TINJAUAN PUSTAKA} \label{Bab II}

\section{Tinjauan Pustaka} \label{II.Tinjauan}
Berisi penelitian terdahulu yang menjadi konsep / pendukung penelitian yang dilakukan. Lakukan pembahasan secara sistematis dengan menjelaskan masalah apa yang diangkat di penelitian terdahulu, metode yang digunakan, kontribusi yang diberikan, serta analisis penulis terkait dengan keunggulan atau keterbatasannya. Tuangkan perbandingan penelitian terdahulu dengan penelitian yang akan dikerjakan, minimal 5 jurnal pembanding (4-5 tahun terakhir). Kemudian penulis sebaiknya melakukan rangkuman terutama terkait dengan peluang pengembangan atau tugas akhir yang akan dilakukan \par

Perujukan literatur dapat dilakukan dengan menambahkan entri baru dalam file \verb|references.bib|. Cara merujuk sitasi menggunakan \verb|\cite{nama label sitasi}|. Hasil sitasi seperti ini: \cite{knuth2001art} atau \cite{Vogels2006Am}. Daftar Pustaka hanya akan memunculkan sitasi yang direferensikan menggunakan command \verb|\cite{}|. \par

Tuliskan \textbf{judul jurnal, penulis jurnal, tahun jurnal terbit, penjelasan isi jurnal, metode penelitian, hasil penelitian \& pengujian}. \par
\begin{enumerate}[noitemsep]
	\item Sistem Informasi Pendaftaran Haji dan Umroh Di PT Multazam Sriwijaya Barakah Palembang Menggunakan Metode Rapid Application Development. \blindtext
	\item Sistem Informasi Umroh Di PT XYZ Lampung Menggunakan Metode Rapid Application Development. \blindtext
\end{enumerate}

Tabel ringkasan tinjauan pustaka ditulis setelah penjelasan masing-masing jurnal. \par
\begin{longtable}{| b{0.05\textwidth}|p{0.2\textwidth}|p{0.2\textwidth}|p{0.15\textwidth}|p{0.25\textwidth}|} % Longtable berguna agar tabel dapat terpotong di halaman baru
	\caption{Literasi Penelitian Terdahulu}
	\label{table:2.literasi}\\
	\hline
	\textbf{No.} & \textbf{Judul} & \textbf{Masalah} & \textbf{Metode} & \textbf{Hasil} \\
	\hline
	\endhead % Agar semua baris diatas ini diulang jika melewati halaman baru (repeat header row)
	1. & Sistem Informasi Pendaftaran Haji dan Umroh Di PT Multazam Sriwijaya Barakah Palembang Menggunakan Metode Rapid Application Development & Belum adanya sistem untuk pendaftaran haji \& umrah & Rapid Application Development & Sistem Informasi Pendaftaran Haji dan Umroh di PT Multazam Sriwijaya Barakah Palembang\\ 
	\hline
	2. & Sistem Informasi Umroh Di PT XYZ Lampung Menggunakan Metode Rapid Application Development & Belum adanya sistem untuk pendaftaran haji \& umrah & Rapid Application Development & Sistem Informasi Pendaftaran Umroh di PT XYZ Lampung\\ 
	\hline
	3. & Sistem Informasi Umroh Di PT XYZ Lampung Menggunakan Metode Rapid Application Development & Belum adanya sistem untuk pendaftaran haji \& umrah & Rapid Application Development & Sistem Informasi Pendaftaran Umroh di PT XYZ Lampung\\ 
	\hline
\end{longtable}

\section{Dasar Teori} \label{II.Teori}
Berisi teori/konsep yang berkaitan/digunakan dalam tugas akhir yang dikerjakan. Gunakanlah data melalui buku/jurnal referensi, publikasi tugas akhir, penelitian, buku, dan informasi web yang dapat dipertanggungjawabkan, hindari penggunaan dasar teori melalui tautan Wikipedia, surat kabar, atau portal berita, yang dapat memiliki isi yang tidak bersifat fakta. \par

\subsection{Teori 1} \label{II.Teori1}
Berikut adalah contoh penyisipan tabel menggunakan \verb|\begin{longtable}{}|: \par
	
	\begin{longtable}{|c|c|c|c|}
		\caption{Contoh Tabel}
		\label{table:2.contoh}\\
		\hline
		Col1 & Col2 & Col2 & Col3 \\
		\hline
		\endhead
		1 & 6 & 87837 & 787 \\ 
		\hline
		2 & 7 & 78 & 5415 \\
		\hline
		3 & 545 & 778 & 7507 \\
		\hline
		4 & 545 & 18744 & 7560 \\
		\hline
		5 & 88 & 788 & 6344 \\
		\hline
	\end{longtable}

\subsubsection{Subsubbab} \label{II.Teori1.1}
Berikut adalah contoh subsubbab. Ini adalah level subbab maksimal dalam laporan Tugas Akhir, dan tidak boleh lebih dalam. \par

Gambar \ref{fig:2.contoh} adalah contoh Gambar yang diambil dari internet yang harus dicantumkan sumbernya dan memiliki lisensi Creative Common. Jika gambar adalah milik peneliti lain atau tidak dibuat atau diambil sendiri maka peneliti wajib meminta izin kepada peneliti lain tersebut untuk mencantumkan gambar. Gunakan \verb|\begin{figure}| untuk memasukkan gambar. Gunakan \verb|\caption{[nama caption]}| untuk memberikan caption gambar. Nomor caption akan diurutkan secara otomatis. Jangan lupa untuk melabeli setiap gambar dengan \verb|\label{[nama label]}|, agar bisa direferensi menggunakan \verb|\ref{[nama label]}| \par
\begin{figure}[H] % Kalau menggunakan H, posisi gambar akan tepat dibawah teks
	\centering
	\includegraphics[width=0.6\textwidth]{figure/keyboard.jpg}
	\caption{Contoh gambar dan caption}
	\label{fig:2.contoh}
	{\footnotesize Sumber: Contoh} % Untuk memberikan sumber
\end{figure}

\subsection{Teori 2} \label{II.teori2}
Untuk membuat sebuah rumus persamaan, gunakan kode \verb|\begin{equationcaptioned}| seperti dibawah: \par
	
\begin{equationcaptioned}[eq:2.sederhana]{
	x + 1 = 2
}{
	Rumus sederhana % Caption rumus
}
\end{equationcaptioned}

Teks caption rumus tidak akan muncul di teks, tetapi akan muncul di Daftar Rumus. \par

\begin{equationcaptioned}[eq:2.mae]{
    MAE = \frac{1}{n} \sum_{i=1}^{n} \left| y_i - \hat{y}_i \right|
}{
    Mean Absolute Error (MAE)
}
\end{equationcaptioned}

Berikut adalah contoh penulisan persamaan yang lebih kompleks, yaitu persamaan distribusi normal. \par

\begin{equationcaptioned}[eq:2.mae]{
		P(x) = \frac{1}{{\sigma \sqrt {2\pi } }}e^{{{ - \left( {x - \mu } \right)^2 } \mathord{\left/ {\vphantom {{ - \left( {x - \mu } \right)^2 } {2\sigma ^2 }}} \right. \kern-\nulldelimiterspace} {2\sigma ^2 }}}
	}{
		Distribusi Normal
	}
\end{equationcaptioned}

Jika menuliskan banyak persamaan secara berurutan, gunakan  \verb|\begin{split}|: \par

\begin{equationcaptioned}[eq:2.mae]{
		\begin{split} 
			2x - 5y &=  8 \\ 
			3x + 9y &=  -12
		\end{split}
	}{
		Sistem persamaan linier
	}
\end{equationcaptioned}
    \newpage
\chapter{Analisis dan Perancangan} \label{Bab III}

\section{Alur Penelitian} \label{III.Alur}
Digambarkan terkait bagaimana proses yang dilakukan dalam penelitian, dari awal sampai dengan akhir. Gambarkan dalam bentuk diagram alir (\textit{flowchart}). \par

\section{Penjabaran Langkah Penelitian} \label{III.Jabar Alur}
Penjelasan detail dari langkah-langkah alur penelitian, yang sudah tergambar dalam flowchart di subbab \ref{III.Alur}. Subsubbab berikut harus sesuai dengan jumlah alur penelitian. \par

\subsection{Langkah 1} \label{III.Langkah 1}
Penjelasan Langkah 1. \par

\subsection{Langkah 2} \label{III.Langkah 2}
Penjelasan Langkah 2. \par

\section{Alat dan Bahan Tugas Akhir} \label{III.Alat dan Bahan}
Berisi alat-alat dan bahan-bahan yang digunakan dalam penelitian. \par

\subsection{Alat} \label{III.Alat}
Alat yang digunakan untuk melakukan penelitian, dapat berupa computer, PC, Arduino, raspberry, etc. Contoh: \par
\begin{enumerate}[noitemsep]
	\item \textit{Notebook} dengan spesifikasi minumum sistem operasi Windows 11, processor AMD Ryzen 5 7430 CPU @ 6 core/2,3 GHz, RAM 16GB DDR4, grafis AMD Radeon RX Vega 7 2GB, SSD 512 GB.
	\item \textit{Smartphone} dengan spesifikasi OS Android OS 12, CPU Snapdragon 778G Octa-core, GPU Adreno 642L, memori 128 GB, RAM 6 GB.
	\item Platform game engine Godot v4.3
	\item Code editor Microsoft Visual Studio Code
	\item Github
\end{enumerate}

\subsection{Bahan} \label{III.Bahan}
Bahan yang digunakan/diperlukan untuk melakukan penelitian, dapat berupa: \par
\begin{enumerate}[noitemsep]
	\item Dataset pihak lain yang diperoleh dengan izin atau dalam lisensi yang diizinkan untuk digunakan secara langsung,
	\item Dataset pihak pertama yang disusun sendiri melalui quisioner, observasi, atau interview,
	\item Dokumen panduan yang mengacu pada standar, hasil tugas akhir, atau artikel yang disitasi dan digunakan. 
\end{enumerate}

\section{Metode Pengembangan/Pengukuran} \label{III.Metode}
Membahas mengenai metode yang digunakan dalam penelitian, berdasarkan dasar teori yang sebelumnya sudah dijelaskan pada subbab \ref{II.Teori}. Setiap Tugas Akhir wajib memiliki metode dalam pelaksanaannya yang sesuai dengan penelitian yang dikerjakan: \par
\begin{enumerate}[noitemsep]
	\item Alur pengembangan tugas akhir, menggunakan flowchart
	\item Cara pengumpulan data yang digunakan ()Kuesioner, Wawancara, pengujian, dan lainnya)
	\item Metode pengembangan tugas akhir (Metode Waterfall, Agile, RAD, dan lainnya).
\end{enumerate}

\section{Ilustrasi Metode Pengembangan/Pengukuran} \label{III.Ilustrasi}
Jelaskan contoh perhitungan dari metode pengemubangan bagi penelitian Tugas Akhir yang menggunakan algoritma perhitungan tertentu. Tidak perlu harus menggunakan seluruh dataset, cukup menggunakan sampel data. Tujuannya untuk menggambarkan alur perhitungan metode dari data awal sampai luaran yang ditargetkan. \par

\section{Rancangan Pengujian} \label{III.Rancang Uji}
Penjabaran terkait rancangan pengujian pada penelitian. Dapat berupa pengujian perangkat keras, lunak, fungsional, dan non-fungsional.
    \newpage
\chapter{HASIL DAN PEMBAHASAN} \label{Bab IV}

\section{Hasil Penelitian} \label{IV.Hasil}
Berisi hasil penelitian berdasarkan rancangan yang sudah dijelaskan pada Bab \ref{Bab III}, terutama dari Subbab \ref{III.Metode}. Bagi yang membuat alat, jelaskan alat yang jadi dalam bentuk apa. Bagi yang membuat aplikasi, jelaskan aplikasi yang jadi dalam bentuk seperti apa. Jabarkan dalam bentuk pseudocode dan dijelaskan per bagian kodenya. Gunakan gambar dan tabel sebagai alat bantu menjelaskan hasil. \par

Contoh implementasi kode dapat ditulis menggunakan \verb|\begin{lstlisting}|. Contoh kode dapat dilihat pada Kode \ref{code:4.contoh}. \par
% Menulis blok kode
\begin{lstlisting}[caption={Akuisisi Gambar}, label={code:4.contoh}]
def process_dataset(dataset_path):
	image_files = glob(os.path.join(dataset_path, '*.png'))
	image_files.sort()
	for image_file in image_files:
		frame = cv2.imread(image_file)
		if frame is None:
			continue
		frame_rgb = cv2.cvtColor(frame, cv2.COLOR_BGR2RGB)
		cv2.imshow('Frame', frame)
		if cv2.waitKey(1) & 0xFF == ord('q'):
			break
	cv2.destroyAllWindows()
def main():
	datasets = get_all_dataset_folders(DATASET_ROOT)
	for dataset in datasets:
		process_dataset(dataset)
		print("print string")
\end{lstlisting}

\section{Hasil Pengujian} \label{IV.Hasil_Uji}
Berikan hasil pengujian berdasarkan rancangan \& skenario yang sudah direncanakan sebelumnya pada Subbab \ref{III.Rancang_Uji}. \par

\begin{longtable}{|c|c|c|c|c|c|c|c|c|}
	\caption{Data \textit{dummy} Pengujian}
	\label{table:4.dummy}\\
	\hline
	\multirow{2}{*}{\textbf{Subjek}} & \multicolumn{7}{|c|}{\textbf{Hasil Prediksi (BPM)}} & \multirow{2}{*}{\textbf{GT}} \\ \cline{2-8}
	& \textbf{F} & \textbf{NA} & \textbf{NO} & \textbf{RC} & \textbf{LC} & \textbf{M} & \textbf{C} & \\ 
	\hline
	\endfirsthead
	\hline
	\multirow{2}{*}{\textbf{Subjek}} & \multicolumn{7}{|c|}{\textbf{Hasil Prediksi (BPM)}} & \multirow{2}{*}{\textbf{GT}} \\ \cline{2-8}
	& \textbf{F} & \textbf{NA} & \textbf{NO} & \textbf{RC} & \textbf{LC} & \textbf{M} & \textbf{C} & \\ 
	\hline
	\endhead
	\hline
	\endfoot
	\hline
	\endlastfoot
	1 & 68 & 69 & 68 & 70 & 68 & 71 & 69 & 68 \\ 
	\hline
	2 & 69 & 69 & 68 & 70 & 68 & 71 & 69 & 69 \\
	\hline
	3 & 70 & 70 & 69 & 71 & 68 & 73 & 69 & 70\\
	\hline
	4 & 71 & 70 & 70 & 72 & 69 & 73 & 70 & 71 \\
	\hline
	5 & 72 & 72 & 70 & 72 & 70 & 74 & 70 & 72 \\
\end{longtable}

\begin{figure}[H]
	\centering
	\includegraphics[width=0.7\textwidth]{figure/zeta.png}
	\caption{Contoh Graf Pengujian}
	\label{fig:4.graf}
\end{figure}

\section{Analisis Hasil Penelitian} \label{IV.Analisis}
Berikan analisis hasil penelitian \& pengujian, berupa data yang didapatkan dari penelitian \& pengujian Tugas Akhir yang sudah anda kerjakan. Gunakan gambar dan tabel sebagai alat bantu menjelaskan analisis hasil. Data luaran penelitian yang dapat dianalisis berupa: \par
\begin{enumerate}[noitemsep]
	\item Hasil pengujian
	\item Hasil kuesioner
	\item Aplikasi yang dikembangkan
\end{enumerate}
Analisis dapat membandingkan dengan hasil penelitian sebelumnya yang memiliki kemiripan topik. \par
    \newpage
\chapter{Kesimpulan dan Saran}

\section{Kesimpulan}
Berisi kesimpulan dari hasil dan pembahasan terkait penelitian yang dilakukan, dapat juga berupa temuan yang Anda dapatkan setelah melakukan penelitian atau analisis terhadap tugas akhir Anda. Berhubungan dengan poin pada rumusan masalah dan tujuan. 

\section{Saran}
Berisi saran mengenai aspek tugas akhir atau temuan yang dapat dikembangkan dan diperkaya di tugas akhir selanjutnya. 
    %----------------------------------------------------------------%

    % Daftar Pustaka
    \newpage
    \phantomsection% 
    \addcontentsline{toc}{chapter}{DAFTAR PUSTAKA}
    \printbibliography[title={Daftar Pustaka}]

    % Lampiran
    % TODO: Tabel Lampiran
    \newpage
    \appendix
    \addcontentsline{toc}{chapter}{LAMPIRAN}
    \chapter*{Lampiran}
    \renewcommand\thesection{\Alph{section}}
    \section{Dataset}

\section{Hasil Wawancara}

\section{Rincian Kasus Uji}

\end{document}

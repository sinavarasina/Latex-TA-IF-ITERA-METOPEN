\newpage
\pagestyle{fancy}
\fancyhf{}
\fancyhead[R]{\thepage}
\chapter{PENDAHULUAN} \label{Bab I}

\section{Latar Belakang} \label{I.Latar Belakang}
Latar Belakang berisi dasar pemikiran, kebutuhan atau alasan yang menjadi ide dari topik tugas akhir. Tujuan utamanya adalah untuk memberikan informasi secukupnya kepada pembaca agar memahami topik yang akan dibahas. Terdapat dua hal yang wajib dikemukakan: \par
\begin{enumerate}[noitemsep]
	\item Deskripsi yang luas dan longgar yang berkaitan dengan bidang/masalah di masyarakat, industry dan atau bidang-bidang lainnya. Deskripsi ini mewakili bidang/masalah secara umum yang berkaitan dengan Teknik Informatika, bekerja dan akan terlibat di dalamnya. Sangat disarankan di sini, sebisa mungkin tidak ada Batasan tentang pilihan teknologi yang akan digunakan. Contoh: bidang transportasi, bidang telekomunikasi, bidang Pendidikan, bidang manufaktur, bidang renewable energi, pariwisata, militer, transportasi, kesehatan, pertanian, pengelolaan infrastruktur dan sebagainya.
	\item Deskripsi lebih khusus dan mendetail yang didapatkan dari poin 1 di atas. Dari deskripsi umum di atas, selanjutkan fokuskan pada fenomena masalah yang akan diangkat. Pendetailan harus mampu membawa masalah kepada masalah yang mennjukkan peran Anda dalam penelitian 
\end{enumerate}
Gambar \ref{fig:1.contoh} adalah contoh Gambar yang diambil dari internet yang harus dicantumkan sumbernya dan memiliki lisensi Creative Common. Jika Gambar adalah milik peneliti lain atau tidak dibuat atau diambil sendiri maka peneliti wajib meminta izin kepada peneliti lain tersebut untuk mencantumkan gambar. 
\begin{figure}[H] % Kalau menggunakan H, posisi gambar akan tepat dibawah teks
	\centering
	\includegraphics[width=0.6\textwidth]{figure/keyboard.jpg}
	\caption{Contoh gambar dan caption}
	\label{fig:1.contoh}
	{\footnotesize Sumber: Contoh}
\end{figure}

\section{Rumusan Masalah} \label{I.Rumusan Masalah}

Berdasarkan latar belakang yang telah diuraikan di atas, maka permasalahan penelitian dirumuskan sebagai berikut: \par

\begin{enumerate}[noitemsep]
	\item Bagaimana
	\item Bagaimana 
\end{enumerate}


\section{Tujuan Penelitian} \label{I.Tujuan}
Berdasarkan rumusan masalah yang telah diuraikan di atas, maka tujuan dari penelitian ini adalah: \par

\begin{enumerate}[noitemsep]
	\item Menentukan 
	\item Mengimplementasikan
\end{enumerate}


\section{Batasan Masalah} \label{I.Batasan}
Adapun batasan masalah dari penelitian ini agar sesuai dengan yang diharapkan adalah sebagai berikut: \par

\begin{enumerate}[noitemsep]
    \item Bahasa pemrograman yang digunakan adalah bahasa pemrograman Python.
    \item 
\end{enumerate}


\section{Manfaat Penelitian} \label{I.Manfaat}
Adapun manfaat yang diperoleh dari hasil penelitian ini adalah sebagai berikut: \par

\begin{enumerate}[noitemsep]
    \item Menghasilkan sistem 
    \item 
\end{enumerate}


\section{Sistematika Penulisan} \label{I.Sistematika}
Sistematika penulisan berisi pembahasan apa yang akan ditulis disetiap Bab. Sistematika pada umumnya berupa paragraf yang setiap paragraf mencerminkan bahasan setiap Bab. \par

\noindent\textbf{Bab I}

Bab ini berisikan penjelasan latar belakang dari topik penelitian yang berlangsung, rumusan masalah dari masalah yang dihadapi pada penjelasan di latar belakang, tujuan dari penelitian, batasan dari penelitian, manfaat dari hasil penelitian, dan sistematika penulisan tugas akhir. \par

\noindent\textbf{Bab II}

Bab ini membahas mengenai teori-teori dan penelitian yang berkaitan dengan penelitian ini.

\noindent\textbf{Bab III}

Bab ini berisikan penjelasan alur kerja sistem, alat dan data yang digunakan, metode yang digunakan, dan rancangan pengujian.

\noindent\textbf{Bab IV}

Bab ini membahas hasil implementasi dan pengujian dari penelitian yang dilakukan.

\noindent\textbf{Bab V}

Bab ini membahas kesimpulan dari hasil penelitian dan juga saran untuk penelitian selanjutnya.
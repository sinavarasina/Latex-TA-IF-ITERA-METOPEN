\newpage
\chapter{PENDAHULUAN} \label{Bab I}

\section{Latar Belakang} \label{I.Latar Belakang}
Perkembangan sistem operasi modern, khususnya berbasis Linux, mendorong kebutuhan akan lingkungan desktop (\textit{desktop environment}) yang lebih efisien, modular, dan dapat beradaptasi dengan teknologi grafis baru seperti Wayland. Wayland merupakan protokol komunikasi antara \textit{compositor} dan aplikasi klien yang menggantikan X11 sebagai standar tampilan modern. Dengan desain yang sederhana, efisien, dan berbasis objek pada tataran protokol, Wayland memberi ruang bagi pengembang untuk membangun sistem tampilan dengan latensi rendah dan arsitektur modular. \par

Dalam praktiknya, ekosistem \textit{desktop environment} arus utama masih bertumpu pada kerangka berparadigma objek atau modular tradisional. GNOME (Mutter/GTK) menggunakan \textit{GObject} sebagai sistem objek di C, KDE Plasma/KWin berbasis Qt (QObject dan \textit{signals/slots}), sedangkan di ranah kompositor Wayland, Sway memanfaatkan \textit{wlroots} (pustaka modular di C). Hyprland ditulis dalam C++ mengintegrasikan Aquamarine sebagai \textit{rendering backend}. Pendekatan tersebut efektif dan matang, namun umumnya belum mengadopsi arsitektur \textit{Entity-Component-System} (ECS). \par

Di sisi lain, paradigma \textit{Data-Oriented Design} (DOD) yang diimplementasikan melalui arsitektur ECS menawarkan penataan data yang lebih bersahabat terhadap \textit{cache} dan paralelisasi, sehingga berpotensi meningkatkan efisiensi jalur eksekusi kritis (misalnya manajemen jendela, komposisi frame, dan penanganan input). Pendekatan ini telah terbukti efektif dalam dunia \textit{game engine} (misal Unity/Bevy), tetapi penerapannya pada sistem \textit{desktop environment} dan kompositor Wayland masih terbatas. \par

Wayland sebagai protokol tidak mengikat arsitektur internal kompositor; karena itu, terdapat peluang penelitian untuk merancang dan mengimplementasikan sebuah \textit{framework} \textit{desktop environment} berbasis ECS di atas Wayland. Framework semacam ini ditujukan untuk mengelola entitas seperti jendela/surface, perangkat input, dan objek render secara efisien, sekaligus menjaga modularitas agar mudah diperluas. Dengan demikian, penelitian ini memposisikan ECS/DOD bukan sebagai klaim pengganti OOP secara umum, melainkan sebagai eksplorasi arsitektur alternatif pada domain \textit{desktop environment} modern. \par

\section{Rumusan Masalah} \label{I.Rumusan Masalah}
Berdasarkan latar belakang tersebut, maka rumusan masalah dalam penelitian ini adalah sebagai berikut: \par
\begin{enumerate}[noitemsep]
    \item Bagaimana merancang arsitektur \textit{framework} \textit{desktop environment} berbasis \textit{Entity-Component-System} (ECS) yang dapat beroperasi di atas protokol Wayland? 
    \item Bagaimana mengimplementasikan \textit{framework} tersebut sehingga mampu mengelola entitas sistem seperti jendela/surface, input, dan rendering secara efisien? 
    \item Bagaimana mengevaluasi performa dan efisiensi \textit{framework} yang dirancang terhadap pemanfaatan sumber daya sistem (memori, latensi input, dan waktu render)? 
\end{enumerate}
\par

\section{Tujuan Penelitian} \label{I.Tujuan}
Tujuan dari penelitian ini adalah sebagai berikut: \par
\begin{enumerate}[noitemsep]
    \item Merancang arsitektur \textit{framework} \textit{desktop environment} berbasis \textit{Entity-Component-System} (ECS) di atas protokol Wayland. 
    \item Mengimplementasikan prototipe \textit{framework} yang mendukung pengelolaan entitas utama seperti jendela/surface, perangkat input, dan sistem rendering. 
    \item Melakukan pengujian performa untuk menganalisis efisiensi dan pemanfaatan sumber daya \textit{framework} yang dikembangkan (memori, latensi input, dan waktu render). 
\end{enumerate}
\par

\section{Batasan Masalah} \label{I.Batasan}
Untuk menjaga ruang lingkup penelitian agar tetap terfokus dan realistis, penelitian ini dibatasi oleh hal-hal berikut: \par
\begin{enumerate}[noitemsep]
    \item \textit{Framework} yang dikembangkan berfokus pada arsitektur dasar \textit{desktop environment}, meliputi manajemen jendela/surface, input, dan rendering, tanpa implementasi penuh fitur desktop seperti panel, \textit{launcher}, atau \textit{file manager}. 
    \item Penelitian ini hanya menggunakan Wayland sebagai \textit{backend} tampilan utama, tanpa dukungan terhadap X11 atau kompositor lain. 
    \item Bahasa pemrograman yang digunakan adalah C++ dengan memanfaatkan pustaka Wayland (\textit{libwayland-server}) dan pustaka grafis OpenGL/Vulkan secara langsung. 
    \item Pengujian performa difokuskan pada efisiensi memori, waktu render, dan tingkat latensi input, bukan pada fitur antarmuka pengguna atau pengalaman pengguna secara keseluruhan. 
\end{enumerate}
\par
Framework ini dapat dikembangkan lebih lanjut dengan integrasi modul \textit{XWayland} untuk kompatibilitas aplikasi berbasis X11, sebagai pengembangan lanjutan di luar lingkup penelitian ini. \par

\section{Manfaat Penelitian} \label{I.Manfaat}
Penelitian ini diharapkan memberikan beberapa manfaat sebagai berikut: \par
\begin{enumerate}[noitemsep]
    \item \textbf{Bagi mahasiswa:} menambah pemahaman tentang penerapan arsitektur \textit{Data-Oriented Design} dan \textit{Entity-Component-System} dalam pengembangan sistem tingkat rendah. 
    \item \textbf{Bagi program studi Teknik Informatika:} menyediakan referensi penelitian di bidang sistem operasi, grafika komputer, dan arsitektur perangkat lunak berbasis DOD/ECS. 
    \item \textbf{Bagi dunia akademik dan penelitian:} memberikan kontribusi terhadap kajian pengembangan \textit{framework} grafis dan \textit{desktop environment} yang lebih efisien dan modular. 
    \item \textbf{Bagi komunitas pengembang \textit{open-source}:} menyediakan landasan arsitektur alternatif yang dapat dikembangkan menjadi kompositor atau \textit{window manager} ringan berbasis Wayland. 
\end{enumerate}
\par

\section{Sistematika Penulisan} \label{I.Sistematika}
\subsection*{Bab I}
Bab ini berisikan penjelasan latar belakang penelitian, rumusan masalah, tujuan, batasan, manfaat penelitian, serta sistematika penulisan tugas akhir. \par

\subsection*{Bab II}
Bab ini membahas tinjauan pustaka dan dasar teori, termasuk konsep Wayland, kompositor, paradigma \textit{Entity-Component-System} (ECS) dan \textit{Data-Oriented Design} (DOD), serta penelitian terdahulu yang relevan. \par

\subsection*{Bab III}
Bab ini menjelaskan metodologi penelitian, perancangan sistem, perangkat lunak dan pustaka yang digunakan, serta rancangan pengujian performa dan efisiensi \textit{framework}. \par

\subsection*{Bab IV}
Bab ini menyajikan hasil implementasi \textit{framework}, pengujian performa, analisis hasil, serta pembahasan mengenai efisiensi dan modularitas sistem yang dikembangkan. \par

\subsection*{Bab V}
Bab ini berisi kesimpulan dari penelitian yang dilakukan serta saran untuk pengembangan lebih lanjut, seperti ekspansi fitur, optimalisasi rendering, dan integrasi dengan komponen \textit{desktop environment} lainnya. \par


